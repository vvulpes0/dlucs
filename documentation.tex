\documentclass[11pt]{book}
\usepackage{amsmath,amssymb}
\usepackage{booktabs}
\usepackage{tabularray}\UseTblrLibrary{booktabs}
\usepackage{Baskervaldx}
\usepackage[margin=1in]{geometry}
\usepackage{enumitem}\setlist[enumerate]{leftmargin=*}
\usepackage{fancyhdr}
\usepackage{multicol}
\usepackage{microtype}
\title{{DLUCS} Lightweight \(\mu\)-Coded System}
\newcommand*{\encoding}[1]{\noindent
\begin{tabular}{|c|c|c|c|c|c|c|c|c|c|c|c|c|c|c|c|}
\multicolumn{1}{c}{15}&
\multicolumn{1}{c}{14}&
\multicolumn{1}{c}{13}&
\multicolumn{1}{c}{12}&
\multicolumn{1}{c}{11}&
\multicolumn{1}{c}{10}&
\multicolumn{1}{c}{9}&
\multicolumn{1}{c}{8}&
\multicolumn{1}{c}{7}&
\multicolumn{1}{c}{6}&
\multicolumn{1}{c}{5}&
\multicolumn{1}{c}{4}&
\multicolumn{1}{c}{3}&
\multicolumn{1}{c}{2}&
\multicolumn{1}{c}{1}&
\multicolumn{1}{c}{0}\\\hline
#1\\\hline
\end{tabular}}
\newcommand*{\instruction}[2][]{%
  \clearpage
  \thispagestyle{fancy}%
  \fancyhf[HL,HR]{\huge{#2}}%
  \fancyhf[HC]{#1}\addtocounter{section}{1}\noindent
}
\let\geq\geqslant
\let\leq\leqslant
\let\emptyset\varnothing
\author{Dakotah Lambert}
\date{}
\begin{document}
\maketitle
\frontmatter
\tableofcontents

\clearpage
\section*{Quick Reference}\markboth{}{}\pagestyle{empty}
\begin{center}
%% \begin{tabular}{clllcccc}
\begin{tblr}{
    colspec={clllcccc},
    stretch=0,
}
  \toprule
  Opcode&&Assembly&Description&C&N&V&Z\\
  \midrule\SetCell[c=8]{c}{\footnotesize\strut
    \textbf{Loads and Stores}}\\\midrule
  00&LD&ld dest,source&Load Direct&--&--&--&--\\
  01&LDI&ld dest,(source)&Load Indirect&--&--&--&--\\
  02&STI&ld (dest),source&Store Indirect&--&--&--&--\\
  &&ldb (dest),source&&&&&\\
  \midrule\SetCell[c=8]{c}{\footnotesize\strut
    \textbf{Arithmetic}}\\\midrule
  03&NEG&neg dest,source&Negate&\(\ast\)&\(\ast\)&\(\ast\)&\(\ast\)\\
  04&SEX&sex dest,source&Sign-Extend Low-Order Byte
  &0&\(\ast\)&0&\(\ast\)\\
  05&OFFS&offs dest,source&Program Counter Offset&--&--&--&--\\
  06&CMP&cmp dest,source&Compare&\(\ast\)&\(\ast\)&\(\ast\)&\(\ast\)\\
  14&ADD&add dest,source&Add&\(\ast\)&\(\ast\)&\(\ast\)&\(\ast\)\\
  15&ADC&adc dest,source&Add With Carry
  &\(\ast\)&\(\ast\)&\(\ast\)&\(\ast\)\\
  16&SUB&sub dest,source&Subtract&\(\ast\)&\(\ast\)&\(\ast\)&\(\ast\)\\
  17&SBC&sbc dest,source&Subtract With Carry
  &\(\ast\)&\(\ast\)&\(\ast\)&\(\ast\)\\
  \midrule\SetCell[c=8]{c}{\footnotesize\strut
    \textbf{Bit Twiddling}}\\\midrule
  08&CLZ&clz dest,source&Count Leading Zeros&--&0&--&\(\ast\)\\
  09&CTZ&ctz dest,source&Count Trailing Zeros&--&0&--&\(\ast\)\\
  0A&CBS&cbs dest,source&Count Bits Set\vphantom{g}&--&0&--&\(\ast\)\\
  \midrule\SetCell[c=8]{c}{\footnotesize\strut
    \textbf{Logic}}\\\midrule
  10&AND&and dest,source&Bitwise Logical AND&--&\(\ast\)&--&\(\ast\)\\
  11&OR&or dest,source&Bitwise Logical Inclusive-OR
  &--&\(\ast\)&--&\(\ast\)\\
  12&XOR&xor dest,source&Bitwise Logical Exclusive-OR
  &--&\(\ast\)&--&\(\ast\)\\
  13&NOT&not dest,source&Logical Complement
  &--&\(\ast\)&--&\(\ast\)\\
  \midrule\SetCell[c=8]{c}{\footnotesize\strut
    \textbf{Rotations and Shifts}}\\\midrule
  18&ROR&ror dest,source&Rotate Right&0&\(\ast\)&0&\(\ast\)\\
  19&ROL&rol dest,source&Rotate Left&0&\(\ast\)&0&\(\ast\)\\
  1A&RRC&rrc dest,source&Rotate Right Through Carry
  &\(\ast\)&\(\ast\)&0&\(\ast\)\\
  1B&RLC&rlc dest,source&Rotate Left Through Carry
  &\(\ast\)&\(\ast\)&0&\(\ast\)\\
  1C&LSR&lsr dest,source&Logical Shift Right
  &\(\ast\)&\(\ast\)&0&\(\ast\)\\
  1D&LSL&lsl dest,source&Logical Shift Left
  &\(\ast\)&\(\ast\)&0&\(\ast\)\\
  1E&ASR&asr dest,source&Arithmetic Shift Right
  &\(\ast\)&\(\ast\)&0&\(\ast\)\\
  \midrule\SetCell[c=8]{c}{\footnotesize\strut
    \textbf{Stack Operations}}\\\midrule
  0C&PUSH&push source&Push to Stack&--&--&--&--\\
  0D&POP&pop dest&Pop From Stack&--&--&--&--\\
  \midrule\SetCell[c=8]{c}{\footnotesize\strut
    \textbf{Control Flow}}\\\midrule
  0E&J&j address&Jump&--&--&--&--\\
  0F&CALL&call address&Jump to Subroutine&--&--&--&--\\
  1F&RET&ret&Return From Subroutine\vphantom{g}&--&--&--&--\\
  \midrule\SetCell[c=8]{c}{\footnotesize\strut
    \textbf{Exchanges}}\\\midrule
  07&XCG&xcg dest,source&Exchange Contents of Registers&--&--&--&--\\
  0B&FLG&flg dest,source&Transfer Flags
  &\(\ast\)&\(\ast\)&\(\ast\)&\(\ast\)\\
  \bottomrule
\end{tblr}
\end{center}

\section*{Condition Codes}\thispagestyle{empty}\pagestyle{empty}
\begin{center}
\begin{tabular}{cclll}
  \toprule
  Code&Name&Test&CMP\\\midrule
  0000&&\(\top\)\\
  0001&c&\(\text{C}\)&\(\geq_U\)\\
  0010&n&\(\text{N}\)\\
  0011&v&\(\text{V}\)\\
  0100&z&\(\text{Z}\)&\(=\)\\
  0101&ls&\(\neg\text{C}\vee\text{Z}\)&\(\leq_U\)\\
  0110&lt&\(\text{N}\oplus\text{V}\)&\(<_S\)\\
  0111&le&\((\text{N}\oplus\text{V})\vee\text{Z}\)
  &\(\leq_S\)\\
  \bottomrule
\end{tabular}\qquad
\begin{tabular}{cclll}
  \toprule
  Code&Name&Test&CMP\\\midrule
  1000&f&\(\bot\)\\
  1001&cc&\(\neg\text{C}\)&\(<_U\)\\
  1010&p&\(\neg\text{N}\)\\
  1011&nv&\(\neg\text{V}\)\\
  1100&nz&\(\neg\text{Z}\)&\(\neq\)\\
  1101&hi&\(\text{C}\wedge\neg\text{Z}\)&\(>_U\)\\
  1110&ge&\(\neg\text{N}\oplus\text{V}\)&\(\geq_S\)\\
  1111&gt&\((\neg\text{N}\oplus\text{V})\wedge\neg\text{Z}\)
  &\(>_S\)\\
  \bottomrule
\end{tabular}
\end{center}

\section*{Registers}
\begin{itemize}
\item 16-bit general purpose registers: R0--R6
\item Stack pointer: SP, aka R7
\item Program counter: PC
\item Flags: Carry (C), Negative (N), Overflow (V), Zero (Z)
\end{itemize}

\noindent
Source and destination fields may always be any
of the general-purpose registers or the stack pointer.
Many instructions also support
16-bit little-endian absolute immediate source operands
following the instruction,
or sign-extended 3-bit absolute immediate source operands
embedded in the instruction.
For rotations and shifts,
3-bit immediate operands are treated as unsigned.

\section*{Additional Assembly Tools}
\begin{itemize}
\item \texttt{\textit{label}:\@ \textit{instruction}} --- assign label
\item \texttt{db [\textit{num} [,\textit{num}\dots]]} ---
  insert bytes (8-bit values) directly
\item \texttt{dw [\textit{num} [,\textit{num}\dots]]} ---
  insert words (16-bit values, little-endian) directly
\item \texttt{RAM} --- switch to RAM segment
\end{itemize}

\section*{Sample Program}
\noindent Compute largest 16-bit unsigned Fibonacci number into R0:

{\noindent\ttfamily\selectfont\small
\null\hspace{0.5in}\hbox to 0.5in{ld\hss}r0,0\\
\null\hspace{0.5in}\hbox to 0.5in{ld\hss}r1,1\\
\null\hbox to 0.5in{loop:\hss}\hbox to 0.5in{add\hss}r0,r1\\
\null\hspace{0.5in}\hbox to 0.5in{xcg\hss}r0,r1\\
\null\hspace{0.5in}\hbox to 0.5in{offs\hss}r6,loop\\
\null\hspace{0.5in}\hbox to 0.5in{offsc\hss}r6,done\\
\null\hbox to 0.5in{done:\hss}\hbox to 0.5in{j\hss}r6}

\mainmatter
\chapter{Architecture Overview}
\section{Introduction}
The \textsc{dlucs} architecture implements
a featureful public instruction set
as microcode atop a simple internal core.
Each instruction consists of a series of up to eight micro-operations,
which allows complex operations to be described by a compact program.

\section{Programmer's Model}
The public instruction set of the \textsc{dlucs} system
is a two-address architecture
intended for use in an educational setting.
Every instruction is conditionally executed.

\subsection{Registers}
There are eight 16-bit registers,
named R0--R6 and SP.
All are general-purpose registers,
in that any may serve as the source or destination
for any instruction,
but SP serves as the stack pointer.
Care should be taken to avoid violating the invariant
that SP points to the top of the stack,
as interrupt processing may use the stack at any time.
Additionally, there is a 16-bit program counter,
and a four-bit flags register.

\subsection{Flags and Conditions}
The four bits stored in the flags register are as follows:
zero (Z), set if the result of an operation is zero,
negative (N), set if the result of an operation has its sign-bit set,
overflow (V), set if an arithmetic operation resulted in signed overflow,
and carry (C), set if an arithmetic operation resulted in carry out,
the unsigned analogue of signed overflow.
These are ordered CNVZ from most to least significant.
Not all operations affect all flags;
see the individual instruction listings for full detail.

Each instruction specifies the conditions
under which it should be executed.
If these conditions are not met,
then the instruction is passed over without effect.

\subsection{Memory System}
The \textsc{dlucs} system is a Princeton architecture system,
with up to 64KB of addressable memory
attached to an external 8-bit bus.
The memory at address zero should be non-volatile,
as this is the initial program counter upon reset.
The stack pointer SP should be placed
such that pushing a value will use writeable memory.

By means of the \textsc{offs} instruction,
indirect addressing, and
the register modes for \textsc{j} and \textsc{call},
the system offers direct support
for writing position-independent code
that will execute properly no matter its location in memory.

\subsection{Initial State}
At reset, the program counter is zero.
No further assumptions may be made.
Namely, the general-purpose registers,
the stack pointer,
and the flags register are in an indeterminate state.

\section{Microprogrammer's Model}
The micro\-programmer sees a smaller, simpler system.
Most work is accomplished through manipulation of two hidden registers.
The microcode control store consists of
24-bit words indexed by the five-bit opcode of the instruction,
a two bit ``mode select'',
and three bits of execution state;
each operation is allowed up to eight micro-operations.
In states 0 and 1, the full instruction has not yet been read
from memory, so these states require extra care.

\subsection{Registers}
The primary registers available to the micro\-programmer are
A and B, which connect directly to the \textsc{alu}.
Additionally, the I register contains the currently latched instruction.
Only a limited selection of general-purpose registers are available,
namely, the stack pointer
and those registers encoded by the source and destination fields
of the instruction.

\subsection{The Bus}
The internal 16-bit data bus may be loaded from one of eight sources:
the constant zero,
the constant one,
the program counter,
the register file
(selecting the source register,
 the destination register,
 or the stack pointer),
the \textsc{alu},
the contents of memory,
the I register, or
the flags register.
For small sources such as the 8-bit memory bus
and the four-bit flags register,
the data is loaded into the least-significant bits,
while the upper bits are filled with zero.
The contents of the bus can be written to any combination of
the A register,
the B register,
the program counter,
memory,
one of the three accessible general-purpose registers,
or the flags register.
By setting a bit in the control word,
the contents of the bus may optionally be byte-swapped.

\subsection{Accessing Memory}
Memory is addressed by either the program counter or the A register.
Either may optionally be post-incremented.
The A register can optionally be post-decremented.

\subsection{ALU Operation}
Seven bits control the \textsc{alu}.
The A and B registers as well as the computed result
may independently be optionally inverted,
with one bit devoted to each.
The remaining four bits determine
which of sixteen operations to perform.
For full detail, see the chapter on microcode operations.

\subsection{Interrupt Handling}
One bit of the control word acknowledges
that handling has begun for a pending interrupt request.


\chapter{Instruction Set Summary}
This chapter provides a brief overview
of the \textsc{dlucs} public instruction set.
Each page is headed with the internal mnemonic and short description
of an instruction.
A mathematical or logical description of the operation is provided,
as well as the standard assembler syntax.
Note that for the indirect load and store instructions,
\textsc{ldi} and \textsc{sti},
the internal mnemonic is not used by the standard assembler.
Instead, these are treated as if they were additional addressing modes
for the \textsc{ld} instruction.
Additionally, each instruction lists its binary format,
an overview of its effect on the flags register,
and listings of the standard microcode implementation.
Micro-operations are described in detail in the following chapter.

The binary formats are given in memory order:
the byte containing the opcode is stored at a lower address
than the byte containing the source
and destination register specifications.
However, when 16-bit immediate data is provided,
this is always treated as a little-endian word.
The least-significant byte is stored at the lower address.
Similarly, return addresses pushed to and popped from the system stack
are interpreted in little-endian fashion.

\section{Condition Codes}
Every \textsc{dlucs} instruction includes a four-bit condition field.
If the condition is not satisfied, then the instruction has no effect
beyond incrementing the program counter.
The available conditions are displayed in the following table,
alongside their respective meanings
if used after a comparison operation.
Conditions are specified as suffixes on instructions.
For instance, ``\texttt{ADD R0,1}''
unconditionally adds 1 into register R0,
while ``\texttt{ADDC R0,1}'' does so
only if the carry flag is set.

\begin{center}
\begin{tabular}{cclll}
  \toprule
  Code&Name&Test&CMP\\\midrule
  0000&&\(\top\)\\
  0001&c&\(\text{C}\)&\(\geq_U\)\\
  0010&n&\(\text{N}\)\\
  0011&v&\(\text{V}\)\\
  0100&z&\(\text{Z}\)&\(=\)\\
  0101&ls&\(\neg\text{C}\vee\text{Z}\)&\(\leq_U\)\\
  0110&lt&\(\text{N}\oplus\text{V}\)&\(<_S\)\\
  0111&le&\((\text{N}\oplus\text{V})\vee\text{Z}\)
  &\(\leq_S\)\\
  \bottomrule
\end{tabular}\qquad
\begin{tabular}{cclll}
  \toprule
  Code&Name&Test&CMP\\\midrule
  1000&f&\(\bot\)\\
  1001&cc&\(\neg\text{C}\)&\(<_U\)\\
  1010&p&\(\neg\text{N}\)\\
  1011&nv&\(\neg\text{V}\)\\
  1100&nz&\(\neg\text{Z}\)&\(\neq\)\\
  1101&hi&\(\text{C}\wedge\neg\text{Z}\)&\(>_U\)\\
  1110&ge&\(\neg\text{N}\oplus\text{V}\)&\(\geq_S\)\\
  1111&gt&\((\neg\text{N}\oplus\text{V})\wedge\neg\text{Z}\)
  &\(>_S\)\\
  \bottomrule
\end{tabular}
\end{center}

\section{Operating Modes}
Most instructions offer three addressing modes.
Mode 0 uses the contents of a register as the source operand.
Mode 1 uses the source field as signed three-bit data,
representing the integers \(-3\) through \(+3\).
The remaining bit pattern, which would correspond to \(-4\),
is used to activate mode 2,
where the source operand is 16-bit little-endian immediate data
following the instruction.
For the shift and rotate instructions,
whose operands are inherently unsigned,
the embedded source field is treated as \emph{unsigned} data.
In this case, mode 2 is activated not by the bit pattern 100,
which validly specifies the number \(4\),
but by the bit pattern 000,
as a zero-bit shift or rotation performs no operation.

The remaining mode pattern, mode 3,
activates when there is a pending interrupt request
upon entering state 0,
and handles this request
by pushing the program counter to the stack
and jumping to address 8.
This request must be acknowledged,
but only at the very last micro-operation.
The standard microcode implementing this is as follows.

{\ttfamily\selectfont\small
  \begin{enumerate}\setcounter{enumi}{-1}
  \item sp \(\rightarrow\) a
  \item 0 \(\rightarrow\) b; a\(-\)
  \item pc \(\rightarrow\) mem; \(\ast\)a|swab|a\(-\)
  \item pc \(\rightarrow\) mem; \(\ast\)a
  \item alu \(\rightarrow\) sp; or
  \item 1 \(\rightarrow\) a,b
  \item alu \(\rightarrow\) a,b; lsl
  \item alu \(\rightarrow\) pc; lsl|ack
  \end{enumerate}
}


\instruction[Add With Carry]{ADC}
\subsection*{Operation}
\(\text{Destination}+\text{Source}+\text{C}
\rightarrow\text{Destination}\)

\subsection*{Assembler Syntax}
\texttt{ADC[\textit{cc}] \textit{destination},\textit{source}}

\subsection*{Description}
Add the source operand to the destination register
along with the carry bit,
and store the result in the destination register.
Valid sources are 16-bit little-endian immediate data,
embedded sign-extended three-bit immediate data,
or a register.

\subsection*{Instruction Encoding}
\encoding{1&0&1&0&1%
&\multicolumn{4}{|c|}{cond}%
&1&\multicolumn{3}{|c|}{rd}&1&0&0}\\\null\qquad
\encoding{\multicolumn{16}{|c|}{immediate data}}

\vspace{2\baselineskip}
\encoding{1&0&1&0&1%
&\multicolumn{4}{|c|}{cond}%
&m&\multicolumn{3}{|c|}{rd}&\multicolumn{3}{|c|}{source}}

\subsection*{Flags}
\begin{multicols}{2}\noindent
  C: set if carry occurred\\
  N: set if result is negative

  \columnbreak\noindent
  V: set if overflow occurred\\
  Z: set if result is zero
\end{multicols}

\subsection*{Standard Microcode}
\begin{multicols}{3}\ttfamily\selectfont\small
  \subsubsection*{Reg--Reg}
  \begin{enumerate}\setcounter{enumi}{-1}
  \item sp \(\rightarrow\) a; pc\(+\)
  \item pc\(+\)
  \item rs \(\rightarrow\) b
  \item rd \(\rightarrow\) a
  \item alu \(\rightarrow\) rd; adc|wf
  \item --
  \item --
  \item --
  \end{enumerate}
  \columnbreak
  \subsubsection*{Short Immediate}
  \begin{enumerate}\setcounter{enumi}{-1}
  \item sp \(\rightarrow\) a; pc\(+\)
  \item pc\(+\)
  \item i \(\rightarrow\) a
  \item alu \(\rightarrow\) b; s3bf
  \item rd \(\rightarrow\) a
  \item alu \(\rightarrow\) rd; adc|wf
  \item --
  \item --
  \end{enumerate}
  \columnbreak
  \subsubsection*{Long Immediate}
  \begin{enumerate}\setcounter{enumi}{-1}
  \item sp \(\rightarrow\) a; pc\(+\)
  \item pc\(+\)
  \item mem \(\rightarrow\) b; pc\(+\)
  \item mem \(\rightarrow\) a; pc\(+\)|swab
  \item alu \(\rightarrow\) b; or
  \item rd \(\rightarrow\) a
  \item alu \(\rightarrow\) rd; adc|wf
  \item --
  \end{enumerate}
\end{multicols}


\instruction[Add]{ADD}
\subsection*{Operation}
\(\text{Destination}+\text{Source}\rightarrow\text{Destination}\)

\subsection*{Assembler Syntax}
\texttt{ADD[\textit{cc}] \textit{destination},\textit{source}}

\subsection*{Description}
Add the source operand to the destination register
and store the result in the destination register.
Valid sources are 16-bit little-endian immediate data,
embedded sign-extended three-bit immediate data,
or a register.

\subsection*{Instruction Encoding}
\encoding{1&0&1&0&0%
&\multicolumn{4}{|c|}{cond}%
&1&\multicolumn{3}{|c|}{rd}&1&0&0}\\\null\qquad
\encoding{\multicolumn{16}{|c|}{immediate data}}

\vspace{2\baselineskip}
\encoding{1&0&1&0&0%
&\multicolumn{4}{|c|}{cond}%
&m&\multicolumn{3}{|c|}{rd}&\multicolumn{3}{|c|}{source}}

\subsection*{Flags}
\begin{multicols}{2}\noindent
  C: set if carry occurred\\
  N: set if result is negative

  \columnbreak\noindent
  V: set if overflow occurred\\
  Z: set if result is zero
\end{multicols}

\subsection*{Standard Microcode}
\begin{multicols}{3}\ttfamily\selectfont\small
  \subsubsection*{Reg--Reg}
  \begin{enumerate}\setcounter{enumi}{-1}
  \item sp \(\rightarrow\) a; pc\(+\)
  \item pc\(+\)
  \item rs \(\rightarrow\) b
  \item rd \(\rightarrow\) a
  \item alu \(\rightarrow\) rd; add|wf
  \item --
  \item --
  \item --
  \end{enumerate}
  \columnbreak
  \subsubsection*{Short Immediate}
  \begin{enumerate}\setcounter{enumi}{-1}
  \item sp \(\rightarrow\) a; pc\(+\)
  \item pc\(+\)
  \item i \(\rightarrow\) a
  \item alu \(\rightarrow\) b; s3bf
  \item rd \(\rightarrow\) a
  \item alu \(\rightarrow\) rd; add|wf
  \item --
  \item --
  \end{enumerate}
  \columnbreak
  \subsubsection*{Long Immediate}
  \begin{enumerate}\setcounter{enumi}{-1}
  \item sp \(\rightarrow\) a; pc\(+\)
  \item pc\(+\)
  \item mem \(\rightarrow\) b; pc\(+\)
  \item mem \(\rightarrow\) a; pc\(+\)|swab
  \item alu \(\rightarrow\) b; or
  \item rd \(\rightarrow\) a
  \item alu \(\rightarrow\) rd; add|wf
  \item --
  \end{enumerate}
\end{multicols}


\instruction[Bitwise Logical And]{AND}
\subsection*{Operation}
\(\text{Destination}\wedge\text{Source}\rightarrow\text{Destination}\)

\subsection*{Assembler Syntax}
\texttt{AND[\textit{cc}] \textit{destination},\textit{source}}

\subsection*{Description}
Compute the bitwise logical \textsc{and}
of the source operand and the destination register,
and store the result in the destination register.
Valid sources are 16-bit little-endian immediate data,
embedded sign-extended three-bit immediate data,
or a register.

\subsection*{Instruction Encoding}
\encoding{1&0&0&0&0%
&\multicolumn{4}{|c|}{cond}%
&1&\multicolumn{3}{|c|}{rd}&1&0&0}\\\null\qquad
\encoding{\multicolumn{16}{|c|}{immediate data}}

\vspace{2\baselineskip}
\encoding{1&0&0&0&0%
&\multicolumn{4}{|c|}{cond}%
&m&\multicolumn{3}{|c|}{rd}&\multicolumn{3}{|c|}{source}}

\subsection*{Flags}
\begin{multicols}{2}\noindent
  C: unaffected\\
  N: set if result is negative

  \columnbreak\noindent
  V: unaffected\\
  Z: set if result is zero
\end{multicols}

\subsection*{Standard Microcode}
\begin{multicols}{3}\ttfamily\selectfont\small
  \subsubsection*{Reg--Reg}
  \begin{enumerate}\setcounter{enumi}{-1}
  \item sp \(\rightarrow\) a; pc\(+\)
  \item pc\(+\)
  \item rs \(\rightarrow\) b
  \item rd \(\rightarrow\) a
  \item alu \(\rightarrow\) rd; and|wf
  \item --
  \item --
  \item --
  \end{enumerate}
  \columnbreak
  \subsubsection*{Short Immediate}
  \begin{enumerate}\setcounter{enumi}{-1}
  \item sp \(\rightarrow\) a; pc\(+\)
  \item pc\(+\)
  \item i \(\rightarrow\) a
  \item alu \(\rightarrow\) b; s3bf
  \item rd \(\rightarrow\) a
  \item alu \(\rightarrow\) rd; and|wf
  \item --
  \item --
  \end{enumerate}
  \columnbreak
  \subsubsection*{Long Immediate}
  \begin{enumerate}\setcounter{enumi}{-1}
  \item sp \(\rightarrow\) a; pc\(+\)
  \item pc\(+\)
  \item mem \(\rightarrow\) b; pc\(+\)
  \item mem \(\rightarrow\) a; pc\(+\)|swab
  \item alu \(\rightarrow\) b; or
  \item rd \(\rightarrow\) a
  \item alu \(\rightarrow\) rd; and|wf
  \item --
  \end{enumerate}
\end{multicols}


\instruction[Arithmetic Shift Right]{ASR}
\subsection*{Operation}
\(\lfloor\text{Destination}/2^{\text{Source}}\rfloor
  \rightarrow\text{Destination}\)

\subsection*{Assembler Syntax}
\texttt{ASR[\textit{cc}] \textit{destination},\textit{source}}

\subsection*{Description}
Shift the destination register right as many times
as specified by the source operand, treated as unsigned.
The most-significant bit of the destination register
is shifted into newly open positions.
Valid sources are 16-bit little-endian immediate data,
embedded unsigned three-bit immediate data,
or a register.

\subsection*{Instruction Encoding}
\encoding{1&1&1&1&0%
&\multicolumn{4}{|c|}{cond}%
&1&\multicolumn{3}{|c|}{rd}&0&0&0}\\\null\qquad
\encoding{\multicolumn{16}{|c|}{immediate data}}

\vspace{2\baselineskip}
\encoding{1&1&1&1&0%
&\multicolumn{4}{|c|}{cond}%
&m&\multicolumn{3}{|c|}{rd}&\multicolumn{3}{|c|}{source}}

\subsection*{Flags}
\begin{multicols}{2}\noindent
  C: set if any set bit was shifted out\\
  N: set if result is negative

  \columnbreak\noindent
  V: 0\\
  Z: set if result is zero
\end{multicols}

\subsection*{Standard Microcode}
\begin{multicols}{3}\ttfamily\selectfont\small
  \subsubsection*{Reg--Reg}
  \begin{enumerate}\setcounter{enumi}{-1}
  \item sp \(\rightarrow\) a; pc\(+\)
  \item 1 \(\rightarrow\) a,b; pc\(+\)
  \item rs \(\rightarrow\) b
  \item rd \(\rightarrow\) a
  \item alu \(\rightarrow\) rd; asr|wf
  \item --
  \item --
  \item --
  \end{enumerate}
  \columnbreak
  \subsubsection*{Short Immediate}
  \begin{enumerate}\setcounter{enumi}{-1}
  \item sp \(\rightarrow\) a; pc\(+\)
  \item 1 \(\rightarrow\) a,b; pc\(+\)
  \item alu \(\rightarrow\) a,b; add+
  \item alu \(\rightarrow\) b; add+
  \item i \(\rightarrow\) a
  \item alu \(\rightarrow\) b; and
  \item rd \(\rightarrow\) a
  \item alu \(\rightarrow\) rd; asr|wf
  \end{enumerate}
  \columnbreak
  \subsubsection*{Long Immediate}
  \begin{enumerate}\setcounter{enumi}{-1}
  \item sp \(\rightarrow\) a; pc\(+\)
  \item 1 \(\rightarrow\) a,b; pc\(+\)
  \item mem \(\rightarrow\) b; pc\(+\)
  \item mem \(\rightarrow\) a; pc\(+\)|swab
  \item alu \(\rightarrow\) b; or
  \item rd \(\rightarrow\) a
  \item alu \(\rightarrow\) rd; asr|wf
  \item --
  \end{enumerate}
\end{multicols}


\instruction[Jump to Subroutine]{CALL}
\subsection*{Operation}
\(\text{PC}\rightarrow\ast({-}{-}\text{SP})\);
\(\text{Source}\rightarrow\text{PC}\)

\subsection*{Assembler Syntax}
\texttt{CALL[\textit{cc}] \textit{address}}

\subsection*{Description}
Push the address of the next instruction onto the stack
and jump to the indicated address.
Valid addresses are 16-bit little-endian immediate data,
or the contents of a register.

\subsection*{Instruction Encoding}
\encoding{0&1&1&1&1%
&\multicolumn{4}{|c|}{cond}%
&1&\multicolumn{3}{|c|}{--}&1&0&0}\\\null\qquad
\encoding{\multicolumn{16}{|c|}{immediate data}}

\vspace{2\baselineskip}
\encoding{0&1&1&1&1%
&\multicolumn{4}{|c|}{cond}%
&0&\multicolumn{3}{|c|}{--}&\multicolumn{3}{|c|}{source}}

\subsection*{Flags}
\begin{multicols}{2}\noindent
  C: unaffected\\
  N: unaffected

  \columnbreak\noindent
  V: unaffected\\
  Z: unaffected
\end{multicols}

\subsection*{Standard Microcode}
\begin{multicols}{2}\ttfamily\selectfont\small
  \subsubsection*{Register}
  \begin{enumerate}\setcounter{enumi}{-1}
  \item sp \(\rightarrow\) a; pc\(+\)
  \item 0 \(\rightarrow\) b; a\(-\)|pc\(+\)
  \item pc \(\rightarrow\) mem; \(\ast\)a|swab|a\(-\)
  \item pc \(\rightarrow\) mem; \(\ast\)a
  \item alu \(\rightarrow\) sp; or
  \item rs \(\rightarrow\) pc
  \item --
  \item --
  \end{enumerate}
  \columnbreak
  \subsubsection*{Long Immediate}
  \begin{enumerate}\setcounter{enumi}{-1}
  \item sp \(\rightarrow\) a; pc\(+\)
  \item 0 \(\rightarrow\) b; a\(-\)|pc\(+\)
  \item pc\(+\)2 \(\rightarrow\) mem; \(\ast\)a|swab|a\(-\)
  \item pc\(+\)2 \(\rightarrow\) mem; \(\ast\)a
  \item alu \(\rightarrow\) sp; or
  \item mem \(\rightarrow\) a; pc\(+\)
  \item mem \(\rightarrow\) b; pc\(+\)|swab
  \item alu \(\rightarrow\) pc; or
  \end{enumerate}
\end{multicols}


\instruction[Count Bits Set]{CBS}
\subsection*{Operation}
\(\#\text{Source}\rightarrow\text{Destination}\)

\subsection*{Assembler Syntax}
\texttt{CBS[\textit{cc}] \textit{destination},\textit{source}}

\subsection*{Description}
Count the number of bits set in the source operand,
and store the result in the destination register.
This is also called the ``population count'' or ``Hamming weight''.
Valid sources are 16-bit little-endian immediate data,
embedded sign-extended three-bit immediate data,
or a register.

\subsection*{Instruction Encoding}
\encoding{0&1&0&1&0%
&\multicolumn{4}{|c|}{cond}%
&1&\multicolumn{3}{|c|}{rd}&1&0&0}\\\null\qquad
\encoding{\multicolumn{16}{|c|}{immediate data}}

\vspace{2\baselineskip}
\encoding{0&1&0&1&0%
&\multicolumn{4}{|c|}{cond}%
&m&\multicolumn{3}{|c|}{rd}&\multicolumn{3}{|c|}{source}}

\subsection*{Flags}
\begin{multicols}{2}\noindent
  C: unaffected\\
  N: 0

  \columnbreak\noindent
  V: unaffected\\
  Z: set if result is zero
\end{multicols}

\subsection*{Standard Microcode}
\begin{multicols}{3}\ttfamily\selectfont\small
  \subsubsection*{Reg--Reg}
  \begin{enumerate}\setcounter{enumi}{-1}
  \item sp \(\rightarrow\) a; pc\(+\)
  \item pc\(+\)
  \item rs \(\rightarrow\) a
  \item alu \(\rightarrow\) rd; cbs|wf
  \item --
  \item --
  \item --
  \item --
  \end{enumerate}
  \columnbreak
  \subsubsection*{Short Immediate}
  \begin{enumerate}\setcounter{enumi}{-1}
  \item sp \(\rightarrow\) a; pc\(+\)
  \item pc\(+\)
  \item i \(\rightarrow\) a
  \item alu \(\rightarrow\) a; s3bf
  \item alu \(\rightarrow\) rd; cbs|wf
  \item --
  \item --
  \item --
  \end{enumerate}
  \columnbreak
  \subsubsection*{Long Immediate}
  \begin{enumerate}\setcounter{enumi}{-1}
  \item sp \(\rightarrow\) a; pc\(+\)
  \item pc\(+\)
  \item mem \(\rightarrow\) b; pc\(+\)
  \item mem \(\rightarrow\) a; pc\(+\)|swab
  \item alu \(\rightarrow\) a; or
  \item alu \(\rightarrow\) rd; cbs|wf
  \item --
  \item --
  \end{enumerate}
\end{multicols}


\instruction[Count Leading Zeros]{CLZ}
\subsection*{Operation}
\(15-\lfloor\log_2(\text{Source})\rfloor\rightarrow\text{Destination}\)

\subsection*{Assembler Syntax}
\texttt{CLZ[\textit{cc}] \textit{destination},\textit{source}}

\subsection*{Description}
Count the number of leading zero-bits in the source operand,
and store the result in the destination register.
Valid sources are 16-bit little-endian immediate data,
embedded sign-extended three-bit immediate data,
or a register.

\subsection*{Instruction Encoding}
\encoding{0&1&0&0&0%
&\multicolumn{4}{|c|}{cond}%
&1&\multicolumn{3}{|c|}{rd}&1&0&0}\\\null\qquad
\encoding{\multicolumn{16}{|c|}{immediate data}}

\vspace{2\baselineskip}
\encoding{0&1&0&0&0%
&\multicolumn{4}{|c|}{cond}%
&m&\multicolumn{3}{|c|}{rd}&\multicolumn{3}{|c|}{source}}

\subsection*{Flags}
\begin{multicols}{2}\noindent
  C: unaffected\\
  N: 0

  \columnbreak\noindent
  V: unaffected\\
  Z: set if result is zero
\end{multicols}

\subsection*{Standard Microcode}
\begin{multicols}{3}\ttfamily\selectfont\small
  \subsubsection*{Reg--Reg}
  \begin{enumerate}\setcounter{enumi}{-1}
  \item sp \(\rightarrow\) a; pc\(+\)
  \item pc\(+\)
  \item rs \(\rightarrow\) a
  \item alu \(\rightarrow\) a; \(\neg\)fd
  \item alu \(\rightarrow\) rd; cbs|wf
  \item --
  \item --
  \item --
  \end{enumerate}
  \columnbreak
  \subsubsection*{Short Immediate}
  \begin{enumerate}\setcounter{enumi}{-1}
  \item sp \(\rightarrow\) a; pc\(+\)
  \item pc\(+\)
  \item i \(\rightarrow\) a
  \item alu \(\rightarrow\) a; s3bf
  \item alu \(\rightarrow\) a; \(\neg\)fd
  \item alu \(\rightarrow\) rd; cbs|wf
  \item --
  \item --
  \end{enumerate}
  \columnbreak
  \subsubsection*{Long Immediate}
  \begin{enumerate}\setcounter{enumi}{-1}
  \item sp \(\rightarrow\) a; pc\(+\)
  \item pc\(+\)
  \item mem \(\rightarrow\) b; pc\(+\)
  \item mem \(\rightarrow\) a; pc\(+\)|swab
  \item alu \(\rightarrow\) a; or
  \item alu \(\rightarrow\) a; \(\neg\)fd
  \item alu \(\rightarrow\) rd; cbs|wf
  \item --
  \end{enumerate}
\end{multicols}


\instruction[Compare]{CMP}
\subsection*{Operation}
\(\text{Destination}-\text{Source}\)

\subsection*{Assembler Syntax}
\texttt{CMP[\textit{cc}] \textit{destination},\textit{source}}

\subsection*{Description}
Subtract the source operand from the destination register.
Do not store the result, but do update flags accordingly.
Valid sources are 16-bit little-endian immediate data,
embedded sign-extended three-bit immediate data,
or a register.

\subsection*{Instruction Encoding}
\encoding{0&0&1&1&0%
&\multicolumn{4}{|c|}{cond}%
&1&\multicolumn{3}{|c|}{rd}&1&0&0}\\\null\qquad
\encoding{\multicolumn{16}{|c|}{immediate data}}

\vspace{2\baselineskip}
\encoding{0&0&1&1&0%
&\multicolumn{4}{|c|}{cond}%
&m&\multicolumn{3}{|c|}{rd}&\multicolumn{3}{|c|}{source}}

\subsection*{Flags}
\begin{multicols}{2}\noindent
  C: set if carry occurred\\
  N: set if result is negative

  \columnbreak\noindent
  V: set if overflow occurred\\
  Z: set if result is zero
\end{multicols}

\subsection*{Standard Microcode}
\begin{multicols}{3}\ttfamily\selectfont\small
  \subsubsection*{Reg--Reg}
  \begin{enumerate}\setcounter{enumi}{-1}
  \item sp \(\rightarrow\) a; pc\(+\)
  \item pc\(+\)
  \item rs \(\rightarrow\) b
  \item rd \(\rightarrow\) a
  \item alu \(\rightarrow\) \(\emptyset\); sub|wf
  \item --
  \item --
  \item --
  \end{enumerate}
  \columnbreak
  \subsubsection*{Short Immediate}
  \begin{enumerate}\setcounter{enumi}{-1}
  \item sp \(\rightarrow\) a; pc\(+\)
  \item pc\(+\)
  \item i \(\rightarrow\) a
  \item alu \(\rightarrow\) b; s3bf
  \item rd \(\rightarrow\) a
  \item alu \(\rightarrow\) \(\emptyset\); sub|wf
  \item --
  \item --
  \end{enumerate}
  \columnbreak
  \subsubsection*{Long Immediate}
  \begin{enumerate}\setcounter{enumi}{-1}
  \item sp \(\rightarrow\) a; pc\(+\)
  \item pc\(+\)
  \item mem \(\rightarrow\) b; pc\(+\)
  \item mem \(\rightarrow\) a; pc\(+\)|swab
  \item alu \(\rightarrow\) b; or
  \item rd \(\rightarrow\) a
  \item alu \(\rightarrow\) \(\emptyset\); sub|wf
  \item --
  \end{enumerate}
\end{multicols}


\instruction[Count Trailing Zeros]{CTZ}
\subsection*{Operation}
\(\lfloor\log_2(\text{Source}\wedge-\text{Source})\rfloor
\rightarrow\text{Destination}\)

\subsection*{Assembler Syntax}
\texttt{CTZ[\textit{cc}] \textit{destination},\textit{source}}

\subsection*{Description}
Count the number of trailing zero-bits in the source operand,
and store the result in the destination register.
Valid sources are 16-bit little-endian immediate data,
embedded sign-extended three-bit immediate data,
or a register.

\subsection*{Instruction Encoding}
\encoding{0&1&0&0&1%
&\multicolumn{4}{|c|}{cond}%
&1&\multicolumn{3}{|c|}{rd}&1&0&0}\\\null\qquad
\encoding{\multicolumn{16}{|c|}{immediate data}}

\vspace{2\baselineskip}
\encoding{0&1&0&0&1%
&\multicolumn{4}{|c|}{cond}%
&m&\multicolumn{3}{|c|}{rd}&\multicolumn{3}{|c|}{source}}

\subsection*{Flags}
\begin{multicols}{2}\noindent
  C: unaffected\\
  N: 0

  \columnbreak\noindent
  V: unaffected\\
  Z: set if result is zero
\end{multicols}

\subsection*{Standard Microcode}
\begin{multicols}{3}\ttfamily\selectfont\small
  \subsubsection*{Reg--Reg}
  \begin{enumerate}\setcounter{enumi}{-1}
  \item sp \(\rightarrow\) a; pc\(+\)
  \item pc\(+\)
  \item rs \(\rightarrow\) a
  \item alu \(\rightarrow\) a; rev
  \item alu \(\rightarrow\) a; \(\neg\)fd
  \item alu \(\rightarrow\) rd; cbs|wf
  \item --
  \item --
  \end{enumerate}
  \columnbreak
  \subsubsection*{Short Immediate}
  \begin{enumerate}\setcounter{enumi}{-1}
  \item sp \(\rightarrow\) a; pc\(+\)
  \item pc\(+\)
  \item i \(\rightarrow\) a
  \item alu \(\rightarrow\) a; s3bf
  \item alu \(\rightarrow\) a; rev
  \item alu \(\rightarrow\) a; \(\neg\)fd
  \item alu \(\rightarrow\) rd; cbs|wf
  \item --
  \end{enumerate}
  \columnbreak
  \subsubsection*{Long Immediate}
  \begin{enumerate}\setcounter{enumi}{-1}
  \item sp \(\rightarrow\) a; pc\(+\)
  \item pc\(+\)
  \item mem \(\rightarrow\) b; pc\(+\)
  \item mem \(\rightarrow\) a; pc\(+\)|swab
  \item alu \(\rightarrow\) a; or
  \item alu \(\rightarrow\) a; rev
  \item alu \(\rightarrow\) a; \(\neg\)fd
  \item alu \(\rightarrow\) rd; cbs|wf
  \end{enumerate}
\end{multicols}


\instruction[Transfer Flags]{FLG}
\subsection*{Operation}
\(\text{Source},\text{Flags}\rightarrow\text{Flags},\text{Destination}\)

\subsection*{Assembler Syntax}
\texttt{FLG[\textit{cc}] \textit{destination},\textit{source}}

\subsection*{Description}
Store the current flags in the destination register
and set the flags to the contents of the source operand.
Valid sources are 16-bit little-endian immediate data
(of which only the low four bits are used),
or a register.
Embedded immediates are not available.

\subsection*{Instruction Encoding}
\encoding{0&1&0&1&1%
&\multicolumn{4}{|c|}{cond}%
&1&\multicolumn{3}{|c|}{rd}&1&0&0}\\\null\qquad
\encoding{\multicolumn{16}{|c|}{immediate data}}

\vspace{2\baselineskip}
\encoding{0&1&0&1&1%
&\multicolumn{4}{|c|}{cond}%
&0&\multicolumn{3}{|c|}{rd}&\multicolumn{3}{|c|}{source}}

\subsection*{Flags}
\begin{multicols}{2}\noindent
  C: set from source\\
  N: set from source

  \columnbreak\noindent
  V: set from source\\
  Z: set from source
\end{multicols}

\subsection*{Standard Microcode}
\begin{multicols}{2}\ttfamily\selectfont\small
  \subsubsection*{Reg--Reg}
  \begin{enumerate}\setcounter{enumi}{-1}
  \item sp \(\rightarrow\) a; pc\(+\)
  \item pc\(+\)
  \item rs \(\rightarrow\) b
  \item flags \(\rightarrow\) a
  \item alu \(\rightarrow\) a; rev
  \item alu \(\rightarrow\) rd; rev
  \item alu \(\rightarrow\) \(\emptyset\); s3bf|wf
  \item --
  \end{enumerate}
  \columnbreak
  \subsubsection*{Long Immediate}
  \begin{enumerate}\setcounter{enumi}{-1}
  \item sp \(\rightarrow\) a; pc\(+\)
  \item pc\(+\)
  \item mem \(\rightarrow\) b; pc\(+\)
  \item flags \(\rightarrow\) a; pc\(+\)
  \item alu \(\rightarrow\) a; rev
  \item alu \(\rightarrow\) rd; rev
  \item alu \(\rightarrow\) \(\emptyset\); s3bf|wf
  \item --
  \end{enumerate}
\end{multicols}


\instruction[Jump]{J}
\subsection*{Operation}
\(\text{Source}\rightarrow\text{PC}\)

\subsection*{Assembler Syntax}
\texttt{J[\textit{cc}] \textit{address}}

\subsection*{Description}
Continue execution at the indicated address.
Valid addresses are 16-bit little-endian immediate data,
or the contents of a register.

\subsection*{Instruction Encoding}
\encoding{0&1&1&1&0%
&\multicolumn{4}{|c|}{cond}%
&1&\multicolumn{3}{|c|}{--}&1&0&0}\\\null\qquad
\encoding{\multicolumn{16}{|c|}{immediate data}}

\vspace{2\baselineskip}
\encoding{0&1&1&1&0%
&\multicolumn{4}{|c|}{cond}%
&0&\multicolumn{3}{|c|}{--}&\multicolumn{3}{|c|}{source}}

\subsection*{Flags}
\begin{multicols}{2}\noindent
  C: unaffected\\
  N: unaffected

  \columnbreak\noindent
  V: unaffected\\
  Z: unaffected
\end{multicols}

\subsection*{Standard Microcode}
\begin{multicols}{2}\ttfamily\selectfont\small
  \subsubsection*{Register}
  \begin{enumerate}\setcounter{enumi}{-1}
  \item sp \(\rightarrow\) a; pc\(+\)
  \item pc\(+\)
  \item rs \(\rightarrow\) pc
  \item --
  \item --
  \item --
  \item --
  \item --
  \end{enumerate}
  \columnbreak
  \subsubsection*{Long Immediate}
  \begin{enumerate}\setcounter{enumi}{-1}
  \item sp \(\rightarrow\) a; pc\(+\)
  \item pc\(+\)
  \item mem \(\rightarrow\) a; pc\(+\)
  \item mem \(\rightarrow\) b; pc\(+\)|swab
  \item alu \(\rightarrow\) pc; or
  \item --
  \item --
  \item --
  \end{enumerate}
\end{multicols}


\instruction[Load]{LD}
\subsection*{Operation}
\(\text{Source}\rightarrow\text{Destination}\)

\subsection*{Assembler Syntax}
\texttt{LD[\textit{cc}] \textit{destination},\textit{source}}

\subsection*{Description}
Copy the contents of the source operand into the destination register.
Valid sources are 16-bit little-endian immediate data,
embedded sign-extended three-bit immediate data,
or a register.

\subsection*{Instruction Encoding}
\encoding{0&0&0&0&0%
&\multicolumn{4}{|c|}{cond}%
&1&\multicolumn{3}{|c|}{rd}&1&0&0}\\\null\qquad
\encoding{\multicolumn{16}{|c|}{immediate data}}

\vspace{2\baselineskip}
\encoding{0&0&0&0&0%
&\multicolumn{4}{|c|}{cond}%
&m&\multicolumn{3}{|c|}{rd}&\multicolumn{3}{|c|}{source}}

\subsection*{Flags}
\begin{multicols}{2}\noindent
  C: unaffected\\
  N: unaffected

  \columnbreak\noindent
  V: unaffected\\
  Z: unaffected
\end{multicols}

\subsection*{Standard Microcode}
\begin{multicols}{3}\ttfamily\selectfont\small
  \subsubsection*{Reg--Reg}
  \begin{enumerate}\setcounter{enumi}{-1}
  \item sp \(\rightarrow\) a; pc\(+\)
  \item pc\(+\)
  \item rs \(\rightarrow\) a,b
  \item alu \(\rightarrow\) rd; or
  \item --
  \item --
  \item --
  \item --
  \end{enumerate}
  \columnbreak
  \subsubsection*{Short Immediate}
  \begin{enumerate}\setcounter{enumi}{-1}
  \item sp \(\rightarrow\) a; pc\(+\)
  \item pc\(+\)
  \item i \(\rightarrow\) a
  \item alu \(\rightarrow\) rd; s3bf
  \item --
  \item --
  \item --
  \item --
  \end{enumerate}
  \columnbreak
  \subsubsection*{Long Immediate}
  \begin{enumerate}\setcounter{enumi}{-1}
  \item sp \(\rightarrow\) a; pc\(+\)
  \item pc\(+\)
  \item mem \(\rightarrow\) b; pc\(+\)
  \item mem \(\rightarrow\) a; pc\(+\)|swab
  \item alu \(\rightarrow\) rd; or
  \item --
  \item --
  \item --
  \end{enumerate}
\end{multicols}


\instruction[Load Indirect]{LDI}
\subsection*{Operation}
\(\ast\text{Source}\rightarrow\text{Destination}\)

\subsection*{Assembler Syntax}
\texttt{LD[\textit{cc}] \textit{destination},(\textit{source})}

\subsection*{Description}
Copy the contents of the memory location
indexed by the source operand into the destination register.
This is always a word-sized transfer.
Valid sources are 16-bit little-endian immediate data
or a register.

\subsection*{Instruction Encoding}
\encoding{0&0&0&0&1%
&\multicolumn{4}{|c|}{cond}%
&1&\multicolumn{3}{|c|}{rd}&1&0&0}\\\null\qquad
\encoding{\multicolumn{16}{|c|}{immediate data}}

\vspace{2\baselineskip}
\encoding{0&0&0&0&1%
&\multicolumn{4}{|c|}{cond}%
&0&\multicolumn{3}{|c|}{rd}&\multicolumn{3}{|c|}{source}}

\subsection*{Flags}
\begin{multicols}{2}\noindent
  C: unaffected\\
  N: unaffected

  \columnbreak\noindent
  V: unaffected\\
  Z: unaffected
\end{multicols}

\subsection*{Standard Microcode}
\begin{multicols}{2}\ttfamily\selectfont\small
  \subsubsection*{Reg--Reg}
  \begin{enumerate}\setcounter{enumi}{-1}
  \item sp \(\rightarrow\) a; pc\(+\)
  \item pc\(+\)
  \item rs \(\rightarrow\) a
  \item mem \(\rightarrow\) b; \(\ast\)a|a\(+\)
  \item mem \(\rightarrow\) a; \(\ast\)a|swab
  \item alu \(\rightarrow\) rd; or
  \item --
  \item --
  \end{enumerate}
  \columnbreak
  \subsubsection*{Long Immediate}
  \begin{enumerate}\setcounter{enumi}{-1}
  \item sp \(\rightarrow\) a; pc\(+\)
  \item pc\(+\)
  \item mem \(\rightarrow\) b; pc\(+\)
  \item mem \(\rightarrow\) a; pc\(+\)|swab
  \item alu \(\rightarrow\) a; or
  \item mem \(\rightarrow\) b; \(\ast\)a|a\(+\)
  \item mem \(\rightarrow\) a; \(\ast\)a|swab
  \item alu \(\rightarrow\) rd; or
  \end{enumerate}
\end{multicols}


\instruction[Logical Shift Left]{LSL}
\subsection*{Operation}
\(\text{Destination}\cdot 2^{\text{Source}}\rightarrow\text{Destination}\)

\subsection*{Assembler Syntax}
\texttt{LSL[\textit{cc}] \textit{destination},\textit{source}}

\subsection*{Description}
Shift the destination register left as many times
as specified by the source operand, treated as unsigned.
Zero is shifted into newly open positions.
Valid sources are 16-bit little-endian immediate data,
embedded unsigned three-bit immediate data,
or a register.

\subsection*{Instruction Encoding}
\encoding{1&1&1&0&1%
&\multicolumn{4}{|c|}{cond}%
&1&\multicolumn{3}{|c|}{rd}&0&0&0}\\\null\qquad
\encoding{\multicolumn{16}{|c|}{immediate data}}

\vspace{2\baselineskip}
\encoding{1&1&1&0&1%
&\multicolumn{4}{|c|}{cond}%
&m&\multicolumn{3}{|c|}{rd}&\multicolumn{3}{|c|}{source}}

\subsection*{Flags}
\begin{multicols}{2}\noindent
  C: set if any set bit was shifted out\\
  N: set if result is negative

  \columnbreak\noindent
  V: 0\\
  Z: set if result is zero
\end{multicols}

\subsection*{Standard Microcode}
\begin{multicols}{3}\ttfamily\selectfont\small
  \subsubsection*{Reg--Reg}
  \begin{enumerate}\setcounter{enumi}{-1}
  \item sp \(\rightarrow\) a; pc\(+\)
  \item 1 \(\rightarrow\) a,b; pc\(+\)
  \item rs \(\rightarrow\) b
  \item rd \(\rightarrow\) a
  \item alu \(\rightarrow\) rd; lsl|wf
  \item --
  \item --
  \item --
  \end{enumerate}
  \columnbreak
  \subsubsection*{Short Immediate}
  \begin{enumerate}\setcounter{enumi}{-1}
  \item sp \(\rightarrow\) a; pc\(+\)
  \item 1 \(\rightarrow\) a,b; pc\(+\)
  \item alu \(\rightarrow\) a,b; add+
  \item alu \(\rightarrow\) b; add+
  \item i \(\rightarrow\) a
  \item alu \(\rightarrow\) b; and
  \item rd \(\rightarrow\) a
  \item alu \(\rightarrow\) rd; lsl|wf
  \end{enumerate}
  \columnbreak
  \subsubsection*{Long Immediate}
  \begin{enumerate}\setcounter{enumi}{-1}
  \item sp \(\rightarrow\) a; pc\(+\)
  \item 1 \(\rightarrow\) a,b; pc\(+\)
  \item mem \(\rightarrow\) b; pc\(+\)
  \item mem \(\rightarrow\) a; pc\(+\)|swab
  \item alu \(\rightarrow\) b; or
  \item rd \(\rightarrow\) a
  \item alu \(\rightarrow\) rd; lsl|wf
  \item --
  \end{enumerate}
\end{multicols}


\instruction[Logical Shift Right]{LSR}
\subsection*{Operation}
\(\lfloor\text{Destination}/2^{\text{Source}}\rfloor
\rightarrow\text{Destination}\)

\subsection*{Assembler Syntax}
\texttt{LSR[\textit{cc}] \textit{destination},\textit{source}}

\subsection*{Description}
Shift the destination register right as many times
as specified by the source operand, treated as unsigned.
Zero is shifted into newly open positions.
Valid sources are 16-bit little-endian immediate data,
embedded unsigned three-bit immediate data,
or a register.

\subsection*{Instruction Encoding}
\encoding{1&1&1&0&0%
&\multicolumn{4}{|c|}{cond}%
&1&\multicolumn{3}{|c|}{rd}&0&0&0}\\\null\qquad
\encoding{\multicolumn{16}{|c|}{immediate data}}

\vspace{2\baselineskip}
\encoding{1&1&1&0&0%
&\multicolumn{4}{|c|}{cond}%
&m&\multicolumn{3}{|c|}{rd}&\multicolumn{3}{|c|}{source}}

\subsection*{Flags}
\begin{multicols}{2}\noindent
  C: set if any set bit was shifted out\\
  N: set if result is negative

  \columnbreak\noindent
  V: 0\\
  Z: set if result is zero
\end{multicols}

\subsection*{Standard Microcode}
\begin{multicols}{3}\ttfamily\selectfont\small
  \subsubsection*{Reg--Reg}
  \begin{enumerate}\setcounter{enumi}{-1}
  \item sp \(\rightarrow\) a; pc\(+\)
  \item 1 \(\rightarrow\) a,b; pc\(+\)
  \item rs \(\rightarrow\) b
  \item rd \(\rightarrow\) a
  \item alu \(\rightarrow\) rd; lsr|wf
  \item --
  \item --
  \item --
  \end{enumerate}
  \columnbreak
  \subsubsection*{Short Immediate}
  \begin{enumerate}\setcounter{enumi}{-1}
  \item sp \(\rightarrow\) a; pc\(+\)
  \item 1 \(\rightarrow\) a,b; pc\(+\)
  \item alu \(\rightarrow\) a,b; add+
  \item alu \(\rightarrow\) b; add+
  \item i \(\rightarrow\) a
  \item alu \(\rightarrow\) b; and
  \item rd \(\rightarrow\) a
  \item alu \(\rightarrow\) rd; lsr|wf
  \end{enumerate}
  \columnbreak
  \subsubsection*{Long Immediate}
  \begin{enumerate}\setcounter{enumi}{-1}
  \item sp \(\rightarrow\) a; pc\(+\)
  \item 1 \(\rightarrow\) a,b; pc\(+\)
  \item mem \(\rightarrow\) b; pc\(+\)
  \item mem \(\rightarrow\) a; pc\(+\)|swab
  \item alu \(\rightarrow\) b; or
  \item rd \(\rightarrow\) a
  \item alu \(\rightarrow\) rd; lsr|wf
  \item --
  \end{enumerate}
\end{multicols}


\instruction[Negate]{NEG}
\subsection*{Operation}
\(-\text{Source}\rightarrow\text{Destination}\)

\subsection*{Assembler Syntax}
\texttt{NEG[\textit{cc}] \textit{destination},\textit{source}}

\subsection*{Description}
Subtract the source operand from zero
and place the result into the destination register.
Valid sources are 16-bit little-endian immediate data,
embedded unsigned three-bit immediate data,
or a register.

\subsection*{Instruction Encoding}
\encoding{0&0&0&1&1%
&\multicolumn{4}{|c|}{cond}%
&1&\multicolumn{3}{|c|}{rd}&1&0&0}\\\null\qquad
\encoding{\multicolumn{16}{|c|}{immediate data}}

\vspace{2\baselineskip}
\encoding{0&0&0&1&1%
&\multicolumn{4}{|c|}{cond}%
&m&\multicolumn{3}{|c|}{rd}&\multicolumn{3}{|c|}{source}}

\subsection*{Flags}
\begin{multicols}{2}\noindent
  C: set if result is zero\\
  N: set if result is negative

  \columnbreak\noindent
  V: set if overflow occurred\\
  Z: set if result is zero
\end{multicols}

\subsection*{Standard Microcode}
\begin{multicols}{3}\ttfamily\selectfont\small
  \subsubsection*{Reg--Reg}
  \begin{enumerate}\setcounter{enumi}{-1}
  \item sp \(\rightarrow\) a; pc\(+\)
  \item pc\(+\)
  \item rs \(\rightarrow\) b
  \item 0 \(\rightarrow\) a
  \item alu \(\rightarrow\) rd; sub|wf
  \item --
  \item --
  \item --
  \end{enumerate}
  \columnbreak
  \subsubsection*{Short Immediate}
  \begin{enumerate}\setcounter{enumi}{-1}
  \item sp \(\rightarrow\) a; pc\(+\)
  \item pc\(+\)
  \item i \(\rightarrow\) a
  \item alu \(\rightarrow\) b; s3bf
  \item 0 \(\rightarrow\) a
  \item alu \(\rightarrow\) rd; sub|wf
  \item --
  \item --
  \end{enumerate}
  \columnbreak
  \subsubsection*{Long Immediate}
  \begin{enumerate}\setcounter{enumi}{-1}
  \item sp \(\rightarrow\) a; pc\(+\)
  \item pc\(+\)
  \item mem \(\rightarrow\) b; pc\(+\)
  \item mem \(\rightarrow\) a; pc\(+\)|swab
  \item alu \(\rightarrow\) b; or
  \item 0 \(\rightarrow\) a
  \item alu \(\rightarrow\) rd; sub|wf
  \item --
  \end{enumerate}
\end{multicols}


\instruction[Logical Complement]{NOT}
\subsection*{Operation}
\(\neg\text{Source}\rightarrow\text{Destination}\)

\subsection*{Assembler Syntax}
\texttt{NOT[\textit{cc}] \textit{destination},\textit{source}}

\subsection*{Description}
Invert the bits of the source operand
and place the result into the destination register.
Valid sources are 16-bit little-endian immediate data,
embedded unsigned three-bit immediate data,
or a register.

\subsection*{Instruction Encoding}
\encoding{1&0&0&1&1%
&\multicolumn{4}{|c|}{cond}%
&1&\multicolumn{3}{|c|}{rd}&1&0&0}\\\null\qquad
\encoding{\multicolumn{16}{|c|}{immediate data}}

\vspace{2\baselineskip}
\encoding{1&0&0&1&1%
&\multicolumn{4}{|c|}{cond}%
&m&\multicolumn{3}{|c|}{rd}&\multicolumn{3}{|c|}{source}}

\subsection*{Flags}
\begin{multicols}{2}\noindent
  C: unaffected\\
  N: set if result is negative

  \columnbreak\noindent
  V: unaffected\\
  Z: set if result is zero
\end{multicols}

\subsection*{Standard Microcode}
\begin{multicols}{3}\ttfamily\selectfont\small
  \subsubsection*{Reg--Reg}
  \begin{enumerate}\setcounter{enumi}{-1}
  \item sp \(\rightarrow\) a; pc\(+\)
  \item pc\(+\)
  \item rs \(\rightarrow\) a,b
  \item alu \(\rightarrow\) rd; nor|wf
  \item --
  \item --
  \item --
  \item --
  \end{enumerate}
  \columnbreak
  \subsubsection*{Short Immediate}
  \begin{enumerate}\setcounter{enumi}{-1}
  \item sp \(\rightarrow\) a; pc\(+\)
  \item pc\(+\)
  \item i \(\rightarrow\) a
  \item alu \(\rightarrow\) a,b; s3bf
  \item alu \(\rightarrow\) rd; nor|wf
  \item --
  \item --
  \item --
  \end{enumerate}
  \columnbreak
  \subsubsection*{Long Immediate}
  \begin{enumerate}\setcounter{enumi}{-1}
  \item sp \(\rightarrow\) a; pc\(+\)
  \item pc\(+\)
  \item mem \(\rightarrow\) b; pc\(+\)
  \item mem \(\rightarrow\) a; pc\(+\)|swab
  \item alu \(\rightarrow\) rd; nor|wf
  \item --
  \item --
  \item --
  \end{enumerate}
\end{multicols}


\instruction[Program Counter Offset]{OFFS}
\subsection*{Operation}
\(\text{PC}+\text{Source}\rightarrow\text{Destination}\)

\subsection*{Assembler Syntax}
\texttt{OFFS[\textit{cc}] \textit{destination},\textit{source}}

\subsection*{Description}
Add the source operand
to the address of the next instruction
and place the result into the destination register.
Valid sources are 16-bit little-endian immediate data,
embedded sign-extended three-bit immediate data,
or a register.
As instructions are two bytes long,
three-bit immediates are doubled for convenience.
For example, the bit pattern 101 corresponds not to -3 but to -6.

\subsection*{Instruction Encoding}
\encoding{0&0&1&0&1%
&\multicolumn{4}{|c|}{cond}%
&1&\multicolumn{3}{|c|}{rd}&1&0&0}\\\null\qquad
\encoding{\multicolumn{16}{|c|}{immediate data}}

\vspace{2\baselineskip}
\encoding{0&0&1&0&1%
&\multicolumn{4}{|c|}{cond}%
&m&\multicolumn{3}{|c|}{rd}&\multicolumn{3}{|c|}{source}}

\subsection*{Flags}
\begin{multicols}{2}\noindent
  C: unaffected\\
  N: unaffected

  \columnbreak\noindent
  V: unaffected\\
  Z: unaffected
\end{multicols}

\subsection*{Standard Microcode}
\begin{multicols}{3}\ttfamily\selectfont\small
  \subsubsection*{Reg--Reg}
  \begin{enumerate}\setcounter{enumi}{-1}
  \item sp \(\rightarrow\) a; pc\(+\)
  \item pc\(+\)
  \item rs \(\rightarrow\) a
  \item pc \(\rightarrow\) b
  \item alu \(\rightarrow\) rd; add
  \item --
  \item --
  \item --
  \end{enumerate}
  \columnbreak
  \subsubsection*{Short Immediate}
  \begin{enumerate}\setcounter{enumi}{-1}
  \item sp \(\rightarrow\) a; pc\(+\)
  \item pc\(+\)
  \item i \(\rightarrow\) a
  \item alu \(\rightarrow\) a,b; s3bf
  \item alu \(\rightarrow\) a; add
  \item pc \(\rightarrow\) b
  \item alu \(\rightarrow\) rd; add
  \item --
  \end{enumerate}
  \columnbreak
  \subsubsection*{Long Immediate}
  \begin{enumerate}\setcounter{enumi}{-1}
  \item sp \(\rightarrow\) a; pc\(+\)
  \item pc\(+\)
  \item mem \(\rightarrow\) b; pc\(+\)
  \item mem \(\rightarrow\) a; pc\(+\)|swab
  \item alu \(\rightarrow\) a; or
  \item pc \(\rightarrow\) b
  \item alu \(\rightarrow\) rd; add
  \item --
  \end{enumerate}
\end{multicols}


\instruction[Bitwise Logical Inclusive-Or]{OR}
\subsection*{Operation}
\(\text{Destination}\vee\text{Source}\rightarrow\text{Destination}\)

\subsection*{Assembler Syntax}
\texttt{OR[\textit{cc}] \textit{destination},\textit{source}}

\subsection*{Description}
Compute the bitwise logical inclusive-\textsc{or}
of the source operand and the destination register,
and store the result in the destination register.
Valid sources are 16-bit little-endian immediate data,
embedded sign-extended three-bit immediate data,
or a register.

\subsection*{Instruction Encoding}
\encoding{1&0&0&0&1%
&\multicolumn{4}{|c|}{cond}%
&1&\multicolumn{3}{|c|}{rd}&1&0&0}\\\null\qquad
\encoding{\multicolumn{16}{|c|}{immediate data}}

\vspace{2\baselineskip}
\encoding{1&0&0&0&1%
&\multicolumn{4}{|c|}{cond}%
&m&\multicolumn{3}{|c|}{rd}&\multicolumn{3}{|c|}{source}}

\subsection*{Flags}
\begin{multicols}{2}\noindent
  C: unaffected\\
  N: set if result is negative

  \columnbreak\noindent
  V: unaffected\\
  Z: set if result is zero
\end{multicols}

\subsection*{Standard Microcode}
\begin{multicols}{3}\ttfamily\selectfont\small
  \subsubsection*{Reg--Reg}
  \begin{enumerate}\setcounter{enumi}{-1}
  \item sp \(\rightarrow\) a; pc\(+\)
  \item pc\(+\)
  \item rs \(\rightarrow\) b
  \item rd \(\rightarrow\) a
  \item alu \(\rightarrow\) rd; or|wf
  \item --
  \item --
  \item --
  \end{enumerate}
  \columnbreak
  \subsubsection*{Short Immediate}
  \begin{enumerate}\setcounter{enumi}{-1}
  \item sp \(\rightarrow\) a; pc\(+\)
  \item pc\(+\)
  \item i \(\rightarrow\) a
  \item alu \(\rightarrow\) b; s3bf
  \item rd \(\rightarrow\) a
  \item alu \(\rightarrow\) rd; or|wf
  \item --
  \item --
  \end{enumerate}
  \columnbreak
  \subsubsection*{Long Immediate}
  \begin{enumerate}\setcounter{enumi}{-1}
  \item sp \(\rightarrow\) a; pc\(+\)
  \item pc\(+\)
  \item mem \(\rightarrow\) b; pc\(+\)
  \item mem \(\rightarrow\) a; pc\(+\)|swab
  \item alu \(\rightarrow\) b; or
  \item rd \(\rightarrow\) a
  \item alu \(\rightarrow\) rd; or|wf
  \item --
  \end{enumerate}
\end{multicols}


\instruction[Pop From Stack]{POP}
\subsection*{Operation}
\(\ast(\text{SP}{+}{+})\rightarrow\text{Destination}\)

\subsection*{Assembler Syntax}
\texttt{POP[\textit{cc}] \textit{destination}}

\subsection*{Description}
Read a word from the top of the stack,
place this result into the destination register,
and advance the stack pointer.

\subsection*{Instruction Encoding}
\encoding{0&1&1&0&1%
&\multicolumn{4}{|c|}{cond}%
&0&\multicolumn{3}{|c|}{rd}&\multicolumn{3}{|c|}{--}}

\subsection*{Flags}
\begin{multicols}{2}\noindent
  C: unaffected\\
  N: unaffected

  \columnbreak\noindent
  V: unaffected\\
  Z: unaffected
\end{multicols}

\subsection*{Standard Microcode}
{\ttfamily\selectfont\small
  \subsubsection*{Register}
  \begin{enumerate}\setcounter{enumi}{-1}
  \item sp \(\rightarrow\) a; pc\(+\)
  \item pc\(+\)
  \item mem \(\rightarrow\) b; \(\ast\)a|a\(+\)
  \item mem \(\rightarrow\) a; \(\ast\)a|swab
  \item alu \(\rightarrow\) rd; or
  \item sp \(\rightarrow\) a
  \item 1 \(\rightarrow\) b; a\(+\)
  \item alu \(\rightarrow\) sp; add
  \end{enumerate}
}


\instruction[Push to Stack]{PUSH}
\subsection*{Operation}
\(\text{Source}\rightarrow\ast({-}{-}\text{SP})\)

\subsection*{Assembler Syntax}
\texttt{PUSH[\textit{cc}] \textit{source}}

\subsection*{Description}
Decrement the stack pointer,
push the source operand onto the top of the stack.
Only register operands may be used.

\subsection*{Instruction Encoding}
\encoding{0&1&1&0&0%
&\multicolumn{4}{|c|}{cond}%
&0&\multicolumn{3}{|c|}{--}&\multicolumn{3}{|c|}{source}}

\subsection*{Flags}
\begin{multicols}{2}\noindent
  C: unaffected\\
  N: unaffected

  \columnbreak\noindent
  V: unaffected\\
  Z: unaffected
\end{multicols}

\subsection*{Standard Microcode}
{\ttfamily\selectfont\small
  \subsubsection*{Register}
  \begin{enumerate}\setcounter{enumi}{-1}
  \item sp \(\rightarrow\) a; pc\(+\)
  \item 0 \(\rightarrow\) b; a\(-\); pc\(+\)
  \item rs \(\rightarrow\) mem; \(\ast\)a|swab|a\(-\)
  \item rs \(\rightarrow\) mem; \(\ast\)a
  \item alu \(\rightarrow\) sp; or
  \item --
  \item --
  \item --
  \end{enumerate}
}


\instruction[Return From Subroutine]{RET}
\subsection*{Operation}
\(\ast(\text{SP}{+}{+})\rightarrow\text{PC}\)

\subsection*{Assembler Syntax}
\texttt{RET[\textit{cc}]}

\subsection*{Description}
Read a word from the top of the stack,
place this result into the program counter,
and advance the stack pointer.
Continue execution from the popped address.

\subsection*{Instruction Encoding}
\encoding{1&1&1&1&1%
&\multicolumn{4}{|c|}{cond}%
&0&\multicolumn{3}{|c|}{--}&\multicolumn{3}{|c|}{--}}

\subsection*{Flags}
\begin{multicols}{2}\noindent
  C: unaffected\\
  N: unaffected

  \columnbreak\noindent
  V: unaffected\\
  Z: unaffected
\end{multicols}

\subsection*{Standard Microcode}
{\ttfamily\selectfont\small
  \subsubsection*{Zero-Address}
  \begin{enumerate}\setcounter{enumi}{-1}
  \item sp \(\rightarrow\) a; pc\(+\)
  \item pc\(+\)
  \item mem \(\rightarrow\) b; \(\ast\)a|a\(+\)
  \item mem \(\rightarrow\) a; \(\ast\)a|swab
  \item alu \(\rightarrow\) pc; or
  \item sp \(\rightarrow\) a
  \item 1 \(\rightarrow\) b; a\(+\)
  \item alu \(\rightarrow\) sp; add
  \end{enumerate}
}


\instruction[Rotate Left Through Carry]{RLC}
\subsection*{Operation}
\(\text{C:Destination}\circlearrowleft \text{Source}\)

\subsection*{Assembler Syntax}
\texttt{RLC[\textit{cc}] \textit{destination},\textit{source}}

\subsection*{Description}
Rotate the destination register left
through the carry bit as many times
as specified by the source operand, treated as unsigned.
Valid sources are 16-bit little-endian immediate data,
embedded unsigned three-bit immediate data,
or a register.

\subsection*{Instruction Encoding}
\encoding{1&1&0&1&1%
&\multicolumn{4}{|c|}{cond}%
&1&\multicolumn{3}{|c|}{rd}&0&0&0}\\\null\qquad
\encoding{\multicolumn{16}{|c|}{immediate data}}

\vspace{2\baselineskip}
\encoding{1&1&0&1&1%
&\multicolumn{4}{|c|}{cond}%
&m&\multicolumn{3}{|c|}{rd}&\multicolumn{3}{|c|}{source}}

\subsection*{Flags}
\begin{multicols}{2}\noindent
  C: set by rotation\\
  N: set if result is negative

  \columnbreak\noindent
  V: 0\\
  Z: set if result is zero
\end{multicols}

\subsection*{Standard Microcode}
\begin{multicols}{3}\ttfamily\selectfont\small
  \subsubsection*{Reg--Reg}
  \begin{enumerate}\setcounter{enumi}{-1}
  \item sp \(\rightarrow\) a; pc\(+\)
  \item 1 \(\rightarrow\) a,b; pc\(+\)
  \item rs \(\rightarrow\) b
  \item rd \(\rightarrow\) a
  \item alu \(\rightarrow\) rd; rlc|wf
  \item --
  \item --
  \item --
  \end{enumerate}
  \columnbreak
  \subsubsection*{Short Immediate}
  \begin{enumerate}\setcounter{enumi}{-1}
  \item sp \(\rightarrow\) a; pc\(+\)
  \item 1 \(\rightarrow\) a,b; pc\(+\)
  \item alu \(\rightarrow\) a,b; add+
  \item alu \(\rightarrow\) b; add+
  \item i \(\rightarrow\) a
  \item alu \(\rightarrow\) b; and
  \item rd \(\rightarrow\) a
  \item alu \(\rightarrow\) rd; rlc|wf
  \end{enumerate}
  \columnbreak
  \subsubsection*{Long Immediate}
  \begin{enumerate}\setcounter{enumi}{-1}
  \item sp \(\rightarrow\) a; pc\(+\)
  \item 1 \(\rightarrow\) a,b; pc\(+\)
  \item mem \(\rightarrow\) b; pc\(+\)
  \item mem \(\rightarrow\) a; pc\(+\)|swab
  \item alu \(\rightarrow\) b; or
  \item rd \(\rightarrow\) a
  \item alu \(\rightarrow\) rd; rlc|wf
  \item --
  \end{enumerate}
\end{multicols}


\instruction[Rotate Left]{ROL}
\subsection*{Operation}
\(\text{Destination}\circlearrowleft \text{Source}\)

\subsection*{Assembler Syntax}
\texttt{ROL[\textit{cc}] \textit{destination},\textit{source}}

\subsection*{Description}
Rotate the destination register left as many times
as specified by the source operand, treated as unsigned.
Valid sources are 16-bit little-endian immediate data,
embedded unsigned three-bit immediate data,
or a register.

\subsection*{Instruction Encoding}
\encoding{1&1&0&0&1%
&\multicolumn{4}{|c|}{cond}%
&1&\multicolumn{3}{|c|}{rd}&0&0&0}\\\null\qquad
\encoding{\multicolumn{16}{|c|}{immediate data}}

\vspace{2\baselineskip}
\encoding{1&1&0&0&1%
&\multicolumn{4}{|c|}{cond}%
&m&\multicolumn{3}{|c|}{rd}&\multicolumn{3}{|c|}{source}}

\subsection*{Flags}
\begin{multicols}{2}\noindent
  C: 0\\
  N: set if result is negative

  \columnbreak\noindent
  V: 0\\
  Z: set if result is zero
\end{multicols}

\subsection*{Standard Microcode}
\begin{multicols}{3}\ttfamily\selectfont\small
  \subsubsection*{Reg--Reg}
  \begin{enumerate}\setcounter{enumi}{-1}
  \item sp \(\rightarrow\) a; pc\(+\)
  \item 1 \(\rightarrow\) a,b; pc\(+\)
  \item rs \(\rightarrow\) b
  \item rd \(\rightarrow\) a
  \item alu \(\rightarrow\) rd; rol|wf
  \item --
  \item --
  \item --
  \end{enumerate}
  \columnbreak
  \subsubsection*{Short Immediate}
  \begin{enumerate}\setcounter{enumi}{-1}
  \item sp \(\rightarrow\) a; pc\(+\)
  \item 1 \(\rightarrow\) a,b; pc\(+\)
  \item alu \(\rightarrow\) a,b; add+
  \item alu \(\rightarrow\) b; add+
  \item i \(\rightarrow\) a
  \item alu \(\rightarrow\) b; and
  \item rd \(\rightarrow\) a
  \item alu \(\rightarrow\) rd; rol|wf
  \end{enumerate}
  \columnbreak
  \subsubsection*{Long Immediate}
  \begin{enumerate}\setcounter{enumi}{-1}
  \item sp \(\rightarrow\) a; pc\(+\)
  \item 1 \(\rightarrow\) a,b; pc\(+\)
  \item mem \(\rightarrow\) b; pc\(+\)
  \item mem \(\rightarrow\) a; pc\(+\)|swab
  \item alu \(\rightarrow\) b; or
  \item rd \(\rightarrow\) a
  \item alu \(\rightarrow\) rd; rol|wf
  \item --
  \end{enumerate}
\end{multicols}


\instruction[Rotate Right]{ROR}
\subsection*{Operation}
\(\text{Destination}\circlearrowright \text{Source}\)

\subsection*{Assembler Syntax}
\texttt{ROR[\textit{cc}] \textit{destination},\textit{source}}

\subsection*{Description}
Rotate the destination register right as many times
as specified by the source operand, treated as unsigned.
Valid sources are 16-bit little-endian immediate data,
embedded unsigned three-bit immediate data,
or a register.

\subsection*{Instruction Encoding}
\encoding{1&1&0&0&0%
&\multicolumn{4}{|c|}{cond}%
&1&\multicolumn{3}{|c|}{rd}&0&0&0}\\\null\qquad
\encoding{\multicolumn{16}{|c|}{immediate data}}

\vspace{2\baselineskip}
\encoding{1&1&0&0&0%
&\multicolumn{4}{|c|}{cond}%
&m&\multicolumn{3}{|c|}{rd}&\multicolumn{3}{|c|}{source}}

\subsection*{Flags}
\begin{multicols}{2}\noindent
  C: 0\\
  N: set if result is negative

  \columnbreak\noindent
  V: 0\\
  Z: set if result is zero
\end{multicols}

\subsection*{Standard Microcode}
\begin{multicols}{3}\ttfamily\selectfont\small
  \subsubsection*{Reg--Reg}
  \begin{enumerate}\setcounter{enumi}{-1}
  \item sp \(\rightarrow\) a; pc\(+\)
  \item 1 \(\rightarrow\) a,b; pc\(+\)
  \item rs \(\rightarrow\) b
  \item rd \(\rightarrow\) a
  \item alu \(\rightarrow\) rd; ror|wf
  \item --
  \item --
  \item --
  \end{enumerate}
  \columnbreak
  \subsubsection*{Short Immediate}
  \begin{enumerate}\setcounter{enumi}{-1}
  \item sp \(\rightarrow\) a; pc\(+\)
  \item 1 \(\rightarrow\) a,b; pc\(+\)
  \item alu \(\rightarrow\) a,b; add+
  \item alu \(\rightarrow\) b; add+
  \item i \(\rightarrow\) a
  \item alu \(\rightarrow\) b; and
  \item rd \(\rightarrow\) a
  \item alu \(\rightarrow\) rd; ror|wf
  \end{enumerate}
  \columnbreak
  \subsubsection*{Long Immediate}
  \begin{enumerate}\setcounter{enumi}{-1}
  \item sp \(\rightarrow\) a; pc\(+\)
  \item 1 \(\rightarrow\) a,b; pc\(+\)
  \item mem \(\rightarrow\) b; pc\(+\)
  \item mem \(\rightarrow\) a; pc\(+\)|swab
  \item alu \(\rightarrow\) b; or
  \item rd \(\rightarrow\) a
  \item alu \(\rightarrow\) rd; ror|wf
  \item --
  \end{enumerate}
\end{multicols}


\instruction[Rotate Right Through Carry]{RRC}
\subsection*{Operation}
\(\text{C:Destination}\circlearrowright \text{Source}\)

\subsection*{Assembler Syntax}
\texttt{RRC[\textit{cc}] \textit{destination},\textit{source}}

\subsection*{Description}
Rotate the destination register right
through the carry bit as many times
as specified by the source operand, treated as unsigned.
Valid sources are 16-bit little-endian immediate data,
embedded unsigned three-bit immediate data,
or a register.

\subsection*{Instruction Encoding}
\encoding{1&1&0&1&0%
&\multicolumn{4}{|c|}{cond}%
&1&\multicolumn{3}{|c|}{rd}&0&0&0}\\\null\qquad
\encoding{\multicolumn{16}{|c|}{immediate data}}

\vspace{2\baselineskip}
\encoding{1&1&0&1&0%
&\multicolumn{4}{|c|}{cond}%
&m&\multicolumn{3}{|c|}{rd}&\multicolumn{3}{|c|}{source}}

\subsection*{Flags}
\begin{multicols}{2}\noindent
  C: set by rotation\\
  N: set if result is negative

  \columnbreak\noindent
  V: 0\\
  Z: set if result is zero
\end{multicols}

\subsection*{Standard Microcode}
\begin{multicols}{3}\ttfamily\selectfont\small
  \subsubsection*{Reg--Reg}
  \begin{enumerate}\setcounter{enumi}{-1}
  \item sp \(\rightarrow\) a; pc\(+\)
  \item 1 \(\rightarrow\) a,b; pc\(+\)
  \item rs \(\rightarrow\) b
  \item rd \(\rightarrow\) a
  \item alu \(\rightarrow\) rd; rrc|wf
  \item --
  \item --
  \item --
  \end{enumerate}
  \columnbreak
  \subsubsection*{Short Immediate}
  \begin{enumerate}\setcounter{enumi}{-1}
  \item sp \(\rightarrow\) a; pc\(+\)
  \item 1 \(\rightarrow\) a,b; pc\(+\)
  \item alu \(\rightarrow\) a,b; add+
  \item alu \(\rightarrow\) b; add+
  \item i \(\rightarrow\) a
  \item alu \(\rightarrow\) b; and
  \item rd \(\rightarrow\) a
  \item alu \(\rightarrow\) rd; rrc|wf
  \end{enumerate}
  \columnbreak
  \subsubsection*{Long Immediate}
  \begin{enumerate}\setcounter{enumi}{-1}
  \item sp \(\rightarrow\) a; pc\(+\)
  \item 1 \(\rightarrow\) a,b; pc\(+\)
  \item mem \(\rightarrow\) b; pc\(+\)
  \item mem \(\rightarrow\) a; pc\(+\)|swab
  \item alu \(\rightarrow\) b; or
  \item rd \(\rightarrow\) a
  \item alu \(\rightarrow\) rd; rrc|wf
  \item --
  \end{enumerate}
\end{multicols}


\instruction[Subtract With Carry]{SBC}
\subsection*{Operation}
\(\text{Destination}-\text{Source}-\neg\text{C}
\rightarrow\text{Destination}\)

\subsection*{Assembler Syntax}
\texttt{SBC[\textit{cc}] \textit{destination},\textit{source}}

\subsection*{Description}
Subtract the source operand from the destination register,
accounting for the carry bit,
and store the result in the destination register.
Valid sources are 16-bit little-endian immediate data,
embedded sign-extended three-bit immediate data,
or a register.

\subsection*{Instruction Encoding}
\encoding{1&0&1&1&1%
&\multicolumn{4}{|c|}{cond}%
&1&\multicolumn{3}{|c|}{rd}&1&0&0}\\\null\qquad
\encoding{\multicolumn{16}{|c|}{immediate data}}

\vspace{2\baselineskip}
\encoding{1&0&1&1&1%
&\multicolumn{4}{|c|}{cond}%
&m&\multicolumn{3}{|c|}{rd}&\multicolumn{3}{|c|}{source}}

\subsection*{Flags}
\begin{multicols}{2}\noindent
  C: set if carry occurred\\
  N: set if result is negative

  \columnbreak\noindent
  V: set if overflow occurred\\
  Z: set if result is zero
\end{multicols}

\subsection*{Standard Microcode}
\begin{multicols}{3}\ttfamily\selectfont\small
  \subsubsection*{Reg--Reg}
  \begin{enumerate}\setcounter{enumi}{-1}
  \item sp \(\rightarrow\) a; pc\(+\)
  \item pc\(+\)
  \item rs \(\rightarrow\) b
  \item rd \(\rightarrow\) a
  \item alu \(\rightarrow\) rd; sbc|wf
  \item --
  \item --
  \item --
  \end{enumerate}
  \columnbreak
  \subsubsection*{Short Immediate}
  \begin{enumerate}\setcounter{enumi}{-1}
  \item sp \(\rightarrow\) a; pc\(+\)
  \item pc\(+\)
  \item i \(\rightarrow\) a
  \item alu \(\rightarrow\) b; s3bf
  \item rd \(\rightarrow\) a
  \item alu \(\rightarrow\) rd; sbc|wf
  \item --
  \item --
  \end{enumerate}
  \columnbreak
  \subsubsection*{Long Immediate}
  \begin{enumerate}\setcounter{enumi}{-1}
  \item sp \(\rightarrow\) a; pc\(+\)
  \item pc\(+\)
  \item mem \(\rightarrow\) b; pc\(+\)
  \item mem \(\rightarrow\) a; pc\(+\)|swab
  \item alu \(\rightarrow\) b; or
  \item rd \(\rightarrow\) a
  \item alu \(\rightarrow\) rd; sbc|wf
  \item --
  \end{enumerate}
\end{multicols}


\instruction[Sign-Extend Low-Order Byte]{SEX}
\subsection*{Operation}
\(\text{Source\({}_{7:0}\)}\rightarrow\text{Destination}\)

\subsection*{Assembler Syntax}
\texttt{SEX[\textit{cc}] \textit{destination},\textit{source}}

\subsection*{Description}
Replace the high-order byte of the source operand
with copies of its \(2^7\) bit,
and store the result in the destination register.
Valid sources are 16-bit little-endian immediate data,
embedded sign-extended three-bit immediate data,
or a register.

\subsection*{Instruction Encoding}
\encoding{0&0&1&0&0%
&\multicolumn{4}{|c|}{cond}%
&1&\multicolumn{3}{|c|}{rd}&1&0&0}\\\null\qquad
\encoding{\multicolumn{16}{|c|}{immediate data}}

\vspace{2\baselineskip}
\encoding{0&0&1&0&0%
&\multicolumn{4}{|c|}{cond}%
&m&\multicolumn{3}{|c|}{rd}&\multicolumn{3}{|c|}{source}}

\subsection*{Flags}
\begin{multicols}{2}\noindent
  C: 0\\
  N: set if result is negative

  \columnbreak\noindent
  V: 0\\
  Z: set if result is zero
\end{multicols}

\subsection*{Standard Microcode}
\begin{multicols}{3}\ttfamily\selectfont\small
  \subsubsection*{Reg--Reg}
  \begin{enumerate}\setcounter{enumi}{-1}
  \item sp \(\rightarrow\) a; pc\(+\)
  \item 1 \(\rightarrow\) a,b; pc\(+\)
  \item alu \(\rightarrow\) a,b; lsl
  \item alu \(\rightarrow\) b; lsl
  \item rs \(\rightarrow\) a
  \item alu \(\rightarrow\) a; lsl
  \item alu \(\rightarrow\) rd; asr|wf
  \item --
  \end{enumerate}
  \columnbreak
  \subsubsection*{Short Immediate}
  \begin{enumerate}\setcounter{enumi}{-1}
  \item sp \(\rightarrow\) a; pc\(+\)
  \item 1 \(\rightarrow\) a,b; pc\(+\)
  \item i \(\rightarrow\) a
  \item alu \(\rightarrow\) a; s3bf
  \item 0 \(\rightarrow\) b
  \item alu \(\rightarrow\) rd; asr|wf
  \item --
  \item --
  \end{enumerate}
  \columnbreak
  \subsubsection*{Long Immediate}
  \begin{enumerate}\setcounter{enumi}{-1}
  \item sp \(\rightarrow\) a; pc\(+\)
  \item 1 \(\rightarrow\) a,b; pc\(+\)
  \item alu \(\rightarrow\) a,b; lsl
  \item alu \(\rightarrow\) b; lsl
  \item mem \(\rightarrow\) a; pc\(+\)
  \item alu \(\rightarrow\) a; lsl|pc\(+\)
  \item alu \(\rightarrow\) rd; asr|wf
  \item --
  \end{enumerate}
\end{multicols}


\instruction[Store Indirect]{STI}
\subsection*{Operation}
\(\text{Source}\rightarrow\ast\text{Destination}\)

\subsection*{Assembler Syntax}
\texttt{LD[\textit{cc}] (\textit{destination}),\textit{source}}\\
\texttt{LDB[\textit{cc}] (\textit{destination}),\textit{source}}\\

\subsection*{Description}
Copy the contents of the source register
into the memory location indexed by the destination register.
Only register operands may be used.
The \(2^6\) bit, which normally indicates immediate mode,
is a size toggle.
If clear, transfer only a byte.
If set, transfer a full word.

\subsection*{Instruction Encoding}
\encoding{0&0&0&1&0%
&\multicolumn{4}{|c|}{cond}%
&s&\multicolumn{3}{|c|}{rd}&\multicolumn{3}{|c|}{source}}

\subsection*{Flags}
\begin{multicols}{2}\noindent
  C: unaffected\\
  N: unaffected

  \columnbreak\noindent
  V: unaffected\\
  Z: unaffected
\end{multicols}

\subsection*{Standard Microcode}
\begin{multicols}{2}\ttfamily\selectfont\small
  \subsubsection*{Byte}
  \begin{enumerate}\setcounter{enumi}{-1}
  \item sp \(\rightarrow\) a; pc\(+\)
  \item pc\(+\)
  \item rd \(\rightarrow\) a
  \item rs \(\rightarrow\) mem; \(\ast\)a|a\(+\)
  \item --
  \item --
  \item --
  \item --
  \end{enumerate}
  \columnbreak
  \subsubsection*{Word}
  \begin{enumerate}\setcounter{enumi}{-1}
  \item sp \(\rightarrow\) a; pc\(+\)
  \item pc\(+\)
  \item rd \(\rightarrow\) a
  \item rs \(\rightarrow\) mem; \(\ast\)a|a\(+\)
  \item rs \(\rightarrow\) mem; \(\ast\)a|swab
  \item --
  \item --
  \item --
  \end{enumerate}
\end{multicols}


\instruction[Subtract]{SUB}
\subsection*{Operation}
\(\text{Destination}-\text{Source}\rightarrow\text{Destination}\)

\subsection*{Assembler Syntax}
\texttt{SUB[\textit{cc}] \textit{destination},\textit{source}}

\subsection*{Description}
Subtract the source operand from the destination register
and store the result in the destination register.
Valid sources are 16-bit little-endian immediate data,
embedded sign-extended three-bit immediate data,
or a register.

\subsection*{Instruction Encoding}
\encoding{1&0&1&1&0%
&\multicolumn{4}{|c|}{cond}%
&1&\multicolumn{3}{|c|}{rd}&1&0&0}\\\null\qquad
\encoding{\multicolumn{16}{|c|}{immediate data}}

\vspace{2\baselineskip}
\encoding{1&0&1&1&0%
&\multicolumn{4}{|c|}{cond}%
&m&\multicolumn{3}{|c|}{rd}&\multicolumn{3}{|c|}{source}}

\subsection*{Flags}
\begin{multicols}{2}\noindent
  C: set if carry occurred\\
  N: set if result is negative

  \columnbreak\noindent
  V: set if overflow occurred\\
  Z: set if result is zero
\end{multicols}

\subsection*{Standard Microcode}
\begin{multicols}{3}\ttfamily\selectfont\small
  \subsubsection*{Reg--Reg}
  \begin{enumerate}\setcounter{enumi}{-1}
  \item sp \(\rightarrow\) a; pc\(+\)
  \item pc\(+\)
  \item rs \(\rightarrow\) b
  \item rd \(\rightarrow\) a
  \item alu \(\rightarrow\) rd; sub|wf
  \item --
  \item --
  \item --
  \end{enumerate}
  \columnbreak
  \subsubsection*{Short Immediate}
  \begin{enumerate}\setcounter{enumi}{-1}
  \item sp \(\rightarrow\) a; pc\(+\)
  \item pc\(+\)
  \item i \(\rightarrow\) a
  \item alu \(\rightarrow\) b; s3bf
  \item rd \(\rightarrow\) a
  \item alu \(\rightarrow\) rd; sub|wf
  \item --
  \item --
  \end{enumerate}
  \columnbreak
  \subsubsection*{Long Immediate}
  \begin{enumerate}\setcounter{enumi}{-1}
  \item sp \(\rightarrow\) a; pc\(+\)
  \item pc\(+\)
  \item mem \(\rightarrow\) b; pc\(+\)
  \item mem \(\rightarrow\) a; pc\(+\)|swab
  \item alu \(\rightarrow\) b; or
  \item rd \(\rightarrow\) a
  \item alu \(\rightarrow\) rd; sub|wf
  \item --
  \end{enumerate}
\end{multicols}


\instruction[Exchange Contents of Two Registers]{XCG}
\subsection*{Operation}
\(\text{Source},\text{Destination}
\rightarrow\text{Destination},\text{Source}\)

\subsection*{Assembler Syntax}
\texttt{XCG[\textit{cc}] \textit{destination},\textit{source}}

\subsection*{Description}
Swap the contents of the source and destination registers.
Only register operands may be used.

\subsection*{Instruction Encoding}
\encoding{0&0&1&1&1%
&\multicolumn{4}{|c|}{cond}%
&0&\multicolumn{3}{|c|}{rd}&\multicolumn{3}{|c|}{source}}

\subsection*{Flags}
\begin{multicols}{2}\noindent
  C: unaffected\\
  N: unaffected

  \columnbreak\noindent
  V: unaffected\\
  Z: unaffected
\end{multicols}

\subsection*{Standard Microcode}
{\ttfamily\selectfont\small
  \subsubsection*{Reg-Reg}
  \begin{enumerate}\setcounter{enumi}{-1}
  \item sp \(\rightarrow\) a; pc\(+\)
  \item pc\(+\)
  \item rd \(\rightarrow\) a
  \item rs \(\rightarrow\) b
  \item alu \(\rightarrow\) a; xor
  \item alu \(\rightarrow\) rs,b; xor
  \item alu \(\rightarrow\) rd; xor
  \item --
  \end{enumerate}
}


\instruction[Bitwise Logical Exclusive-Or]{XOR}
\subsection*{Operation}
\(\text{Destination}\oplus\text{Source}\rightarrow\text{Destination}\)
\subsection*{Assembler Syntax}
\texttt{XOR[\textit{cc}] \textit{destination},\textit{source}}

\subsection*{Description}
Compute the bitwise logical exclusive-\textsc{or}
of the source operand and the destination register,
and store the result in the destination register.
Valid sources are 16-bit little-endian immediate data,
embedded sign-extended three-bit immediate data,
or a register.

\subsection*{Instruction Encoding}
\encoding{1&0&0&1&0%
&\multicolumn{4}{|c|}{cond}%
&1&\multicolumn{3}{|c|}{rd}&1&0&0}\\\null\qquad
\encoding{\multicolumn{16}{|c|}{immediate data}}

\vspace{2\baselineskip}
\encoding{1&0&0&1&0%
&\multicolumn{4}{|c|}{cond}%
&m&\multicolumn{3}{|c|}{rd}&\multicolumn{3}{|c|}{source}}

\subsection*{Flags}
\begin{multicols}{2}\noindent
  C: unaffected\\
  N: set if result is negative

  \columnbreak\noindent
  V: unaffected\\
  Z: set if result is zero
\end{multicols}

\subsection*{Standard Microcode}
\begin{multicols}{3}\ttfamily\selectfont\small
  \subsubsection*{Reg--Reg}
  \begin{enumerate}\setcounter{enumi}{-1}
  \item sp \(\rightarrow\) a; pc\(+\)
  \item pc\(+\)
  \item rs \(\rightarrow\) b
  \item rd \(\rightarrow\) a
  \item alu \(\rightarrow\) rd; xor|wf
  \item --
  \item --
  \item --
  \end{enumerate}
  \columnbreak
  \subsubsection*{Short Immediate}
  \begin{enumerate}\setcounter{enumi}{-1}
  \item sp \(\rightarrow\) a; pc\(+\)
  \item pc\(+\)
  \item i \(\rightarrow\) a
  \item alu \(\rightarrow\) b; s3bf
  \item rd \(\rightarrow\) a
  \item alu \(\rightarrow\) rd; xor|wf
  \item --
  \item --
  \end{enumerate}
  \columnbreak
  \subsubsection*{Long Immediate}
  \begin{enumerate}\setcounter{enumi}{-1}
  \item sp \(\rightarrow\) a; pc\(+\)
  \item pc\(+\)
  \item mem \(\rightarrow\) b; pc\(+\)
  \item mem \(\rightarrow\) a; pc\(+\)|swab
  \item alu \(\rightarrow\) b; or
  \item rd \(\rightarrow\) a
  \item alu \(\rightarrow\) rd; xor|wf
  \item --
  \end{enumerate}
\end{multicols}


\chapter{Micro-Operation Summary}
\begin{multicols}{2}\noindent
\begin{tabular}{rcl}
  \toprule
  \multicolumn{3}{c}{\textbf{Control Lines}}\\\midrule
  Bits&&Description\\\midrule
  23--21&&Bus source\\
  20&\(\ast\)a&Address using A instead of PC\\
  19&pc\(+\)&Increment PC\\
  18&a\(-\)&Decrement A (if not b17)\\
  17&a\(+\)&Increment A (requires b18)\\
  16&swab&Swap bus low/high bytes\\
  15&ack&Acknowledge interrupt\\
  14--13&&Register select (0=rs,1=sp,2=rd)\\
  12&&Complement A\\
  11&&Complement B\\
  10&&Complement D\\
  9--6&&Operation select\\
  5&pc&Write to PC\\
  4&mem&Write to memory\\
  3&a&Write to A\\
  2&b&Write to B\\
  1&&Write to register (per regsel)\\
  0&wf&Set flags from \textsc{alu}\\\bottomrule
\end{tabular}

\vspace{\baselineskip}\noindent
\begin{tabular}{ccl}
  \toprule
  \multicolumn{3}{c}{\textbf{Sources}}\\\midrule
  Bits&&Description\\\midrule
  000&0&Zero\\
  001&1&One\\
  010&pc&PC if regsel\({}\neq{}\)3\\
  &pc\(+\)2&PC\({}+{}\)2 if regsel\({}={}\)3\\
  011&&Register (per regsel)\\
  100&alu&Output of \textsc{alu}\\
  101&mem&Memory contents\\
  110&i&Instruction\\
  111&flags&Condition codes\\
  \bottomrule
\end{tabular}

\columnbreak\noindent
\begin{tabular}{ccl@{}c@{~}c@{~}c}
  \toprule
  \multicolumn{6}{c}{\textbf{Operations}
      (``\(\neg\)'' toggles \(\neg\)D)}\\\midrule
  Bits&&Description&\(\neg\)A&\(\neg\)B&\(\neg\)D\\\midrule
  0000&nand&A NAND B&&&\\
  &and&A AND B&&&\(\ast\)\\
  &or&A OR B&\(\ast\)&\(\ast\)&\\
  &nor&A NOR B&\(\ast\)&\(\ast\)&\(\ast\)\\
  0001&adc&Add with carry&&&\\
  &sbc&Sub with ``carry''&&\(\ast\)&\\
  0010&add&Add&&&\\
  0011&add+1&Add, carry forced&&&\\
      &sub&Subtract&&\(\ast\)&\\
  0100&xor&A XOR B&&&\\
  0101&lsl&Logical shift L&&&\\
  0110&lsr&Logical shift R&&&\\
  0111&asr&Arithmetic shift R&&&\\
  1000&rol&Rotate L&&&\\
  1001&rlc&Rotate L through C&&&\\
  1010&rol&Rotate R&&&\\
  1011&rlc&Rotate R through C&&&\\
  1100&cbs&Count bits set&&&\\
  1101&fd&Flood down&&&\\
  1110&rev&Reverse&&&\\
  1111&s3bf&Extend 3b, B\(\to\)flags&&&\\
  \bottomrule
\end{tabular}

\vspace{\baselineskip}\noindent
Micro-operations consist of a source followed by an arrow
and then zero or more non-flag destinations
(using \(\emptyset\) for zero),
optionally followed by a semicolon
and a list of additional flags separated by \(|\).
For \(0\rightarrow\emptyset\),
this expression and its following semicolon can be omitted.

\begin{center}
\texttt{source \(\rightarrow\) dest[,dest\dots]; flag[|flag\dots]}
\end{center}

\noindent
If no bits are set, this is the end of the instruction.
\end{multicols}

\section{ALU Operations}
If the \textsc{alu} is output onto the bus,
the result is one of the following.
Note that flags are always computed,
but only written out if the \texttt{wf} bit is set.

\subsection{NAND (Operation 0)}
Compute the bitwise logical \textsc{nand} of registers A and B.
By controlling the three complementation bits,
this also computes the logical
\textsc{and}, \textsc{or}, and \textsc{nor},
among other things.

\begin{multicols}{2}
  \begin{itemize}
  \item C: unaffected
  \item N: set if result is negative
  \item V: unaffected
  \item Z: set if result is zero
  \end{itemize}
\end{multicols}

\subsection{Adder (Operations 1, 2, and 3)}
The ALU contains a 16-bit adder with carry-in and carry-out bits.
In operation 1 (\texttt{adc}),
the flags register supplies the carry bit.
In operation 2 (\texttt{add}), it is forced to zero.
And in operation 3 (\texttt{add+1}), it is forced to one.
That is, operation 3 can be used to compute
A\({}-{}\)B by setting the \(\neg\)B bit,
or B\({}-{}\)A by setting the \(\neg\)A bit.

\begin{multicols}{2}
  \begin{itemize}
  \item C: set if carry-out occurred
  \item N: set if result is negative
  \item V: set if signed overflow occurred
  \item Z: set if result is zero
  \end{itemize}
\end{multicols}

\subsection{Exclusive OR (Operation 4)}
Compute the bitwise logical exclusive-\textsc{or} of registers A and B.
If an odd number of \(\neg\)A, \(\neg\)B, or \(\neg\)D are set,
this becomes the bitwise logical \textsc{xnor}.

\begin{multicols}{2}
  \begin{itemize}
  \item C: unaffected
  \item N: set if result is negative
  \item V: unaffected
  \item Z: set if result is zero
  \end{itemize}
\end{multicols}

\subsection{Shift Unit (Operations 5, 6, and 7)}
Shift the A register's contents left or right
by the value in the B register.
For the logical shifts, operations 5 and 6,
zero is shifted in.
For the arithmetic right shift, operation 7,
the sign bit is propagated down.

\begin{multicols}{2}
  \begin{itemize}
  \item C: set if any 1-bit is shifted out
  \item N: set if result is negative
  \item V: 0
  \item Z: set if result is zero
  \end{itemize}
\end{multicols}

\subsection{Rotator Unit (Operations 8--11)}
Rotate the contents of the A register
by the value in the B register.
For operations 8 and 10,
the register is rotated within itself.
For operations 9 and 11,
the register is rotated through the carry bit.

\begin{multicols}{2}
  \begin{itemize}
  \item C: 0 (8/10) or by rotation (9/11)
  \item N: set if result is negative
  \item V: 0
  \item Z: set if result is zero
  \end{itemize}
\end{multicols}

\subsection{Count Bits Set (Operation 12)}
Count the number of 1-bits in the A register.

\begin{multicols}{2}
  \begin{itemize}
  \item C: unaffected
  \item N: 0
  \item V: unaffected
  \item Z: set if result is zero
  \end{itemize}
\end{multicols}

\subsection{Flood Down (Operation 13)}
Set every bit up to the most-significant set bit
of the contents of the A register.
This computes \(2^{\lfloor\log_2\text{A}\rfloor+1}-1\).

\begin{multicols}{2}
  \begin{itemize}
  \item C: unaffected
  \item N: 0
  \item V: unaffected
  \item Z: set if result is zero
  \end{itemize}
\end{multicols}

\subsection{Reverse Bits (Operation 14)}
Reverse the bits of the contents of the A register.

\begin{multicols}{2}
  \begin{itemize}
  \item C: unaffected
  \item N: set if result is negative
  \item V: unaffected
  \item Z: set if result is zero
  \end{itemize}
\end{multicols}

\subsection{Sign-Extend Low 3 Bits, B to Flags (Operation 15)}
Output a signed value corresponding
to the low three bits of the A register.
Additionally, output the lowest-order bits of the B register
as flags.

\begin{multicols}{2}
  \begin{itemize}
  \item C: set from B
  \item N: set from B
  \item V: set from B
  \item Z: set from B
  \end{itemize}
\end{multicols}

\end{document}
